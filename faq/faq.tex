\documentclass[letterpaper]{article}

\usepackage{html}
\usepackage[latin1]{inputenc}
\pagestyle{plain}

\bodytext{bgcolor="#FFFFFF"}
\begin{document}

\title{FAQ -- Frequently Asked Questions}
\author{}
\date{\today}
\begin{rawhtml}
    <h1>
      ROBOOP - A robotics object oriented package in C++
    </h1>
    <p>
      [ <a href="./">Welcome</a> | 
      <a href="whatsnew.php">What's New</a> | 
      <a href="faq.html">FAQ</a> | 
      <a href="bugrep.php">Bugs</a> | 
      <a href="suggest.php">Suggestions</a> | 
      <a href="download.php">Download</a> | 
      <a href="mailinglists.php">Mailing Lists</a> | 
      <a href="links.php">Links</a> ]
    </p>
  <hr>
\end{rawhtml}
\maketitle

This page contains answers to common questions, along with some tips
and tricks that we have found useful and presented here as questions.

\tableofcontents

\subsection{Is the package using \texttt{float} or \texttt{double} ?}

The package can use the \texttt{Real} type that can be either
\texttt{float} or \texttt{double} for numerical computations. The file
\texttt{include.h} in the directory \texttt{newmat} contains the
following lines:
\begin{verbatim}
#define USING_DOUBLE // elements of type double
//#define USING_FLOAT // elements of type float
\end{verbatim}
that are used to select the floating point type used.

\subsection{Compiling error under \textsf{Visual C++ 6.0}}

When compiling the files under \textsf{Visual C++ 6.0}, it results in
the following error:
\begin{verbatim}
c:\roboop\newmat\newmat.h(1404) : fatal error C1001: INTERNAL COMPILER ERROR
        (compiler file 'msc1.cpp', line 1786)
         Please choose the Technical Support command on the Visual C++ etc ...
\end{verbatim}
You have to apply the \htmladdnormallink{latest service
  pack}{http://msdn2.microsoft.com/en-us/vstudio/aa718364.aspx}
to \textsf{Visual C++ 6.0}.

\subsection{Error with \texttt{roboop.sln} file under \textsf{Visual C++ .NET}}

If you get an error when opening the \texttt{roboop.sln} file, open
the \texttt{roboop.dsw} instead.

\subsection{What is the performance of the package with respect to the
  time required to evaluate the kinematics, dynamics, etc ?}

The program \texttt{bench.cpp} can be used to benchmark some of the
functionalities of the package for a 6 dof PUMA robot model (see the
following table).
\begin{center}
  \begin{tabular}{|l|r|r|r|r|r|}
    \hline
    \multicolumn{6}{|c|}{Computing times (in milliseconds)} \\
    \hline
    Computer & \multicolumn{4}{|c|}{Pentium II 600MHz} & Pentium II 400MHz \\
    \hline
    Compiler/Function & cygwin gcc 3.3.1 & Borland C++ 5.5.1 &
    OpenWatcom C++ 1.2 & Visual C++ .NET & Linux gcc 3.3.2 \\
    \hline
    Forward Kinematics &  0.39 & 0.08 & 0.07 & 0.04 & 0.08 \\
    \hline
    Inverse Kinematics & 10.74 & 2.21 & 1.90 & 1.39 & 2.31 \\
    \hline
    Jacobian &            0.67 & 0.12 & 0.10 & 0.08 & 0.13 \\
    \hline
    Torque &              3.88 & 0.60 & 0.58 & 0.42 & 0.56 \\
    \hline
    Acceleration &       18.16 & 2.85 & 2.68 & 1.96 & 2.76 \\
    \hline
  \end{tabular}
\end{center}

\begin{rawhtml}
  <hr>
    <p>
      [ <a href="./">Welcome</a> | 
      <a href="whatsnew.php">What's New</a> | 
      <a href="faq.html">FAQ</a> | 
      <a href="bugrep.php">Bugs</a> | 
      <a href="suggest.php">Suggestions</a> | 
      <a href="download.php">Download</a> | 
      <a href="mailinglists.php">Mailing Lists</a> | 
      <a href="links.php">Links</a> ]
    </p>
\end{rawhtml}

\end{document}
