\documentclass[11pt,fleqn,letterpaper]{report}
\usepackage{hyperref}
\usepackage{verbatim}

\newcommand{\mbold}[1]{\mbox{\boldmath $ #1 $}}
\newcommand{\matr}[2]{\left[\begin{array}{#1} #2 \end{array}\right]}
\newcommand{\dfdx}[2]{\frac{\partial #1}{\partial #2}}

\begin{document}
\title{\textsf{ROBOOP} \\ A Robotics Object Oriented Package in C++ \\ version \input{version} \\
\ \\ Documentation}

\author{\htmladdnormallink{Richard Gourdeau}{http://www.professeurs.polymtl.ca/richard.gourdeau/} \\ 
D\'epartement de g\'enie \'electrique \\
\'Ecole Polytechnique de Montr\'eal \\ C.P. 6079, Succ. Centre-Ville, \\
Montr\'eal, Qu\'ebec, Canada, H3C 3A7 \\ 
email: \htmladdnormallink{richard.gourdeau@polymtl.ca}{mailto:richard.gourdeau@polymtl.ca}
}

\maketitle 

\tableofcontents

\chapter{Introduction}

\section{Description}

This package (\textsf{ROBOOP}\footnote{Program source and
  documentation are available from the URL: {\tt
    \htmladdnormallink{http://sourceforge.net/projects/roboop/}{http://sourceforge.net/projects/roboop/}}})
is a C++ robotics object oriented programming toolbox suitable for
synthesis, and simulation of robotic manipulator models in an
environment that provides ``\textsf{MATLAB} like'' features for the
treatment of matrices. Its is a portable tool that does not require
the use of commercial software. A class named {\tt Robot} provides the
implementation of the kinematics, the dynamics and the linearized
dynamics of serial robotic manipulators. A class named {\tt Stewart}
provides the implementation of the kinematics, the dynamics for
Stewart type parallel manipulators.

\section{Requirements}

This work uses the matrix library \textsf{NEWMAT11}
\footnote{available from the site {\tt
    \htmladdnormallink{http://www.robertnz.net/}{{http://www.robertnz.net/}}}}
developed by Robert Davies. Hence, the requirement for the
\textsf{ROBOOP} are the same as for the \textsf{NEWMAT11}. Although
make files are only provided for the\textsf{ Borland C++} 4.5 and 5.x,
\textsf{Visual C++} 6.0, \textsf{Visual C++} 7.0 (.NET),
and \textsf{GNU G++} compilers, other
compilers supporting the \textsf{STL} could be used.  See the file
{\tt nm11.htm} in the {\tt newmat} directory for more details.

The library \htmladdnormallink{\textsf{Boost}}{http://www.boost.org/}
is used by \textsf{ROBOOP}. Under most \textsf{Linux} distributions
and \textsf{Cygwin}, \textsf{Boost} is a standard package (just
install it). For Borland C++, \textsf{Visual C++} and \textsf{QNX},
you can copy the header directory \texttt{boost} from the 
(\htmladdnormallink{\textsf{Boost} library}{http://www.boost.org/})
in the \texttt{roboop/source} directory \footnote{simpler but will not
  provide you with all the \textsf{Boost} features}. Under Mac OSX, if
you are using \htmladdnormallink{\textsf{Homebrew}}{http://brew.sh/},
just use the command \texttt{brew install boost}. 
% or install \textsf{Boost} using the instructions provided in the
% \textsf{Boost} web site (the preferred approach if you intend to use
% it in other projects)}.

In order to use the graphic features of this package, the software
\textsf{gnuplot}\footnote{ \textsf{gnuplot} is freely available from
  the following location: {\tt
    \htmladdnormallink{http://www.gnuplot.info/}{http://www.gnuplot.info/}}
  } (version 3.5 on later) must be installed in the {\tt PATH} of your
computer. The binary name is {\tt gnuplot.exe} under \textsf{Windows
  95/98/NT/2000} (Win32) and \texttt{gnuplot} under most of other
platforms, you should edit the file \texttt{gnugraph.h} if the
binary name is different.


\section{Compiling}

\subsection{Linux}

Under \textsf{Linux}, you can compile using one of the three following
ways (in the \texttt{roboop} directory):
\begin{enumerate}
\item Using the command 
\begin{verbatim}
make -f makefile.gcc
\end{verbatim}
\item If you have
  \htmladdnormallink{\textsf{CMake}}{http://www.cmake.org/} installed
  then use
\begin{verbatim}
cmake .
make
\end{verbatim}
\item If you have
  \htmladdnormallink{\textsf{Bakefile}}{http://bakefile.sourceforge.net/}
  installed then use
\begin{verbatim}
bakefile -f gnu roboop.bkl
make
\end{verbatim}
\end{enumerate}

\subsection{MS Windows}

\begin{description}
\item[Borland Compiler] : you can compile using one of the three
  following ways:
  \begin{enumerate}
  \item Using the command 
\begin{verbatim}
make -f makefile.bc5
\end{verbatim}
  \item If you have
    \htmladdnormallink{\textsf{CMake}}{http://www.cmake.org/}
    installed then use the \texttt{CMake} program from the
    \textsf{Start} menu to generate a \textsf{Borland makefile}, then
    from the prompt (in the \texttt{roboop} directory) execute the
    command
\begin{verbatim}
make
\end{verbatim}
  \item If you have
    \htmladdnormallink{\textsf{Bakefile}}{http://bakefile.sourceforge.net/}
    installed then use (in the \texttt{roboop} directory)
\begin{verbatim}
bakefile -f borland roboop.bkl
make
\end{verbatim}
  \end{enumerate}
\item[Cygwin and MinGW]: you can compile using one of the three following
ways (in the \texttt{roboop} directory):
\begin{enumerate}
\item Using the command 
\begin{verbatim}
make -f makefile.gw32
\end{verbatim}
\item If you have
  \htmladdnormallink{\textsf{CMake}}{http://www.cmake.org/} installed
  then use
\begin{verbatim}
cmake .
make
\end{verbatim}
\item If you have
  \htmladdnormallink{\textsf{Bakefile}}{http://bakefile.sourceforge.net/}
  installed then use
\begin{verbatim}
ln -s /usr/include/boost-1_33_1/boost/ /usr/include/boost
bakefile -f gnu roboop.bkl
make
\end{verbatim}
\end{enumerate}
\item[Visual C++]: you can compile using one of the following ways:
  \begin{enumerate}
  \item Using the command 
\begin{verbatim}
nmake -f makefile.vcpp
\end{verbatim}
  \item Opening the \textsf{Visual C++ 6.0 Workspace}
    \texttt{roboop.dsw} or the \textsf{Visual C++ 7.0 Solution}
    \texttt{roboop.sln} and building the targets.
  \item If you have
    \htmladdnormallink{\textsf{CMake}}{http://www.cmake.org/}
    installed then use the \texttt{CMake} program from the
    \textsf{Start} menu to generate \textsf{NMake makefiles}, then
    from the prompt (in the \texttt{roboop} directory) execute the
    command
\begin{verbatim}
nmake
\end{verbatim}
  \item If you have
    \htmladdnormallink{\textsf{CMake}}{http://www.cmake.org/}
    installed then use the \texttt{CMake} program from the
    \textsf{Start} menu to generate one of the different
    \textsf{Visual Studio project} formats available, then by opening
    the \textsf{Visual C++ Workspace or Solution} generated and
    building the targets.
  \item If you have
    \htmladdnormallink{\textsf{Bakefile}}{http://bakefile.sourceforge.net/}
    installed then use (in the \texttt{roboop} directory)
\begin{verbatim}
bakefile -f msvc roboop.bkl
nmake
\end{verbatim}
or
\begin{verbatim}
bakefile -f msvc6proj roboop.bkl
\end{verbatim}
and by opening the \textsf{Visual C++ Workspace} generated and
building the targets.
  \end{enumerate}
\end{description}

\subsection{Mac OSX}

You can compile using one of the following ways (in the
\texttt{roboop} directory):
\begin{enumerate}
\item Using the command 
\begin{verbatim}
make -f makefile.gccOSX
\end{verbatim}
\item If you have
  \htmladdnormallink{\textsf{CMake}}{http://www.cmake.org/} installed
  then use
\begin{verbatim}
cmake .
make
\end{verbatim}
\item If you have
  \htmladdnormallink{\textsf{Bakefile}}{http://bakefile.sourceforge.net/}
  installed then use
\begin{verbatim}
bakefile -f gnu roboop.bkl
make
\end{verbatim}
\end{enumerate}

\subsection{QNX}

Under \textsf{QNX}, you can compile using the command (in the
\texttt{roboop} directory):
\begin{verbatim}
make -f makefile.qnx
\end{verbatim}

\section{Copyright}

ROBOOP -- A robotics object oriented package in C++, \\
Copyright \copyright\ 1996--2004  Richard Gourdeau


This library is free software; you can redistribute it and/or
modify it under the terms of the GNU Lesser General Public
License as published by the Free Software Foundation; either
version 2.1 of the License, or (at your option) any later version.

This library is distributed in the hope that it will be useful,
but WITHOUT ANY WARRANTY; without even the implied warranty of
MERCHANTABILITY or FITNESS FOR A PARTICULAR PURPOSE.  See the GNU
Lesser General Public License for more details.

You should have received a copy of the GNU Lesser General Public
License along with this library (see appendix~\ref{chap:gnulgpl}); if
not, write to the Free Software Foundation, Inc., 59 Temple Place,
Suite 330, Boston, MA 02111-1307 USA



\section{Version history}

\begin{description}

\item[version 1.32] (2013/XX/XX)

  \textsf{OpenWatcom} support is dropped.

  Upgraded the matrix library to \textsf{NEWMAT11 (beta) November
    2008}

  Code clean up dealing with some warnings.

  Call to \textsf{Gnuplot} under \textsf{Windows} is now using pipes.

  Removed \textsf{CVS} keywords tags.

  \texttt{inv\_kin} immobile joint index bug corrected (thanks to
  Matteo Malosio).

\item[version 1.31] (2006/12/14)
  
  The project can now use
  \htmladdnormallink{\textsf{CMake}}{http://www.cmake.org/} or
  \htmladdnormallink{\textsf{Bakefile}}{http://bakefile.sourceforge.net/}
  for automated \texttt{makefile} generation. In future releases,
  \emph{hand made} makefiles and project files will be replaced by the
  output of \htmladdnormallink{\textsf{CMake}}{http://www.cmake.org/}
  or
  \htmladdnormallink{\textsf{Bakefile}}{http://bakefile.sourceforge.net/}.

  Corrected bug in irotk (reported by Chris Lightcap).
  
\item[version 1.30] (2006/08/17)
  
  Upgraded the matrix library to \textsf{NEWMAT11 (beta) April 2006}
  enabling compilation under \textsf{GNU g++ 4.1.x}.
  
\item[version 1.29] (2006/05/19)
  
  \textsf{OpenWatcom} support is (temporally) suspended. Fixed gear
  ratio bug for viscous friction (reported by Carmine Lia). Fix
  \texttt{set\_q}, \texttt{set\_qp} bug in \texttt{xdot} (reported by
  Philip Gruebele)

  The following changes have been contributed by Etienne Lachance
  \begin{itemize}
  \item ``Clean up'' of some header files.
  \item Member functions add and select are now in template form.
  \item Using \textsf{Boost} shared pointers in \texttt{gnugraph}.
  \item The inverse kinematics function (\texttt{inv\_kin}) should
    return the solution without changing the robot position (reported
    by J.D. Yamokoski).
  \item Functions \texttt{Rhino\_DH}, \texttt{Puma\_DH},
    \texttt{Schilling\_DH}, \texttt{Rhino\_mDH}, \texttt{Puma\_mDH}
    and \texttt{Schilling\_mDH} use \texttt{const Robot\_basic}
    reference instead of\texttt{ const Robot\_basic} pointer.
  \item Prevent exceptions from leaving \texttt{Robot\_basic} destructor.
  \item Catch exception by reference instead of by value.
  \end{itemize}
  
\item[version 1.28] (2005/12/07)

  The following changes have been contributed by Etienne Lachance
  \begin{itemize}
  \item Removing unnecessary copy constructor and the assignment
    operator (operator=) in many classes.
  \item In the \texttt{Quaternion} class, the operator* and operator/
    are now non-member functions when one of the operand is a real, it
    now supports q2 = c * q1 and > q2 = q1 * c
  \end{itemize}

\item[version 1.27] (2005/10/11)

  It is now possible to turn off warning messages in the
  \texttt{Config} class.
  
\item[version 1.26] (2005/07/05)
  \begin{itemize}
  \item New Class \texttt{Stewart} contributed by Samuel Belanger
    (intergated by Etienne Lachance and Richard Gourdeau): new files
    \texttt{stewart.h} and \texttt{stewart.cpp} and modified
    \texttt{bench.cpp}.
  \item Fixed max() bug for VC++ 6.0 (\texttt{utils.cpp}).
  \item Typos in Doxygen documentation.
  \end{itemize}
  
\item[version 1.25] (2005/06/13)
  Fixed \texttt{catch(bad\_alloc)} in constructors.
  
  The following changes have been contributed by Etienne Lachance
  \begin{itemize}
  \item The desired joint acceleration was missing in the computed
    torque method (bug reported by Carmine Lia).
  \item Added missing file message in \texttt{trajectory.cpp}
  \end{itemize}
  The following changes have been contributed by Carmine Lia
  \begin{itemize}
  \item Added \texttt{defined(\_\_MINGW32\_\_)} for temp files in
    \texttt{gnugraph.cpp}.
  \item Added \texttt{pinv} in \texttt{utils.cpp}.
  \end{itemize}

\item[version 1.24] (2005/03/18) 
  
  The following changes have been contributed by Brian Galardo,
  Jean-Pascal Joary, Etienne Lachance:
  \begin{itemize}
  \item Added member functions \texttt{Robot::inv\_schilling}, \par
    \texttt{mRobot::inv\_schilling} and
    \texttt{mRobot\_min\_para::inv\_schilling} for the
    \textsf{Schilling Titan II} robot arm,
  \item Fixed previous bug on Rhino and Puma inverse kinematics.
  \end{itemize}
  by Etienne Lachance: 
  \begin{itemize}
  \item Some ``clean-up'' in the \texttt{config.h} and
    \texttt{config.cpp} files,
  \end{itemize}
  and by Stephen Webb :
  \begin{itemize}
  \item minor bug in constructor \texttt{Robot\_basic(const
      Robot\_basic \& x)}.
  \end{itemize}

\item[version 1.23] (2004/09/18) 

  The following change has been contributed by Etienne Lachance: 
  \begin{itemize}
  \item Configuration files can use degrees for the angles with the
    option \texttt{angle\_in\_degree} set to 1.
  \end{itemize}

\item[version 1.22] (2004/09/10) 

  The following change has been contributed by Etienne Lachance:
  \begin{itemize}
  \item In \texttt{config.cpp}: parameter value can now contain space
    and fixed print member function.
  \end{itemize}
  Carl Glen Henshaw provided a makefile for MAC OS X.

\item[version 1.21] (2004/08/16) 

  The following changes have been contributed by Etienne Lachance
  \begin{itemize}
  \item Fixed some missing \texttt{use\_namespace \#define}.
  \item Merge all \texttt{select\_*} and \texttt{add\_*} functions
    into overloaded \texttt{select()} and \texttt{add()} functions.
  \item made \texttt{gnuplot.cpp} and \texttt{config.cpp} independent
    of \texttt{robot.h} and \texttt{utils.h}.
  \item New constructors for \texttt{Robot} and \texttt{mRobot} based
    on input matrices (this change is NOT backward compatible)
  \end{itemize}
  The following changes have been contributed by Ethan Tira-Thompson
  \begin{itemize}
  \item Supports for \texttt{Link::immobile} flag so jacobians and
    deltas are 0 for immobile joints.
  \item Jacobians will only contain entries for mobile joints -
    otherwise NaNs result in later processing.
  \item Added parameters to jacobian functions to generate for frames
    other than the end effector.
  \item Can now do inverse kinematics for frames other than end
    effector.
  \item Tolerance in \texttt{inv\_kin} based on \texttt{USING\_FLOAT}
    from newmat's \texttt{include.h}
  \end{itemize}

\item[version 1.20] (2004/07/02) 

  The following changes have been contributed by Ethan Tira-Thompson

  \begin{itemize}
  \item Added support for newmat's \texttt{use\_namespace}
    \texttt{\#define}, using \texttt{ROBOOP} namespace.
  \item Fixed some problem using \texttt{float} as \texttt{Real} type.
  \end{itemize}

  The following changes have been contributed by Etienne Lachance

  \begin{itemize}
  \item Added the following class: \texttt{Dynamics},
    \texttt{Trajectory\_Select}, \par \texttt{Proportional\_Derivative}
    and\texttt{ Control\_Select}.
  \item Added a new demo program, call \texttt{demo\_2dof\_pd}. This
    new demo program shows how to use the class mentioned above.
  \item Protection added on input vector of the \texttt{trans}
    function.
  \item Added a \texttt{joint\_offset} logic. This idea has been
    proposed by Ethan Tira-Thompson.
  \item Added \textsf{Doxygen} documentation.
  \item Replace files \texttt{impedance.*} by \texttt{controller.*}.
  \end{itemize}


\item[version 1.19] (2004/05/12) Upgraded the matrix library from
  \textsf{NEWMAT10} to \textsf{NEWMAT11 (beta)}. \textsf{Visual C++
    .NET} and \textsf{Borland C++ Builder 6} compilers are now
  supported. Updated documentation.

\item[version 1.18] (2004/05/05) \textsf{ROBOOP} is relicensed to the
  GNU Lesser General Public License. Updated documentation.

  The following changes have been contributed by Vincent Drolet and
  Etienne Lachance:
  \begin{itemize}
  \item Added the following members function in class Robot:
    \texttt{inv\_kin\_rhino}, \texttt{inv\_kin\_puma} and
    \texttt{robotType\_inv\_kin}.
  \end{itemize}

\item[version 1.17] (2004/04/02) Numerous warning messages were
  corrected under \textsf{VC++}. Updated documentation.

  The following changes have been contributed by Etienne Lachance:
  \begin{itemize}
  \item Added class \texttt{Impedance} which implements the impedance
    controller.
  \item Added function \texttt{perturb\_robot}.
  \item Added class \texttt{Resolve\_acc} which implements the resolve
    rate acceleration position controller.
  \item Added class \texttt{Computed\_torque\_method} which implements
    the computed torque method position controller.
  \item Class \emph{Config} can now write data into a configuration
    file.
  \item Fixed bugs in \texttt{Quaternion} class member functions:
    \texttt{exponential} and \texttt{logarithm}.
  \item Added \texttt{Quaternion} class member function
    \texttt{power}.
  \item Added the following \texttt{Quaternion} class non member
    functions: \texttt{Omega}, \texttt{Slerp}, \texttt{Slerp\_prime},
    \texttt{Squad} and \texttt{Squad\_prime}.
  \item Provided \texttt{Spl\_Quaternion} class to generate
    quaternions cubic splines.
  \item Added class \texttt{Spl\_Cubic} to generate cubic splines.
  \item Added class \texttt{Spl\_path} to generate 3D cubic splines.
  \item Provided \texttt{CLIK} class for closed loop inverse
    kinematics.
  \item Added member functions \texttt{G} and \texttt{C} in all robot
    classes.
  \end{itemize}

\item[version 1.16] (2003/09/24) The
  \htmladdnormallink{\textsf{OpenWatcom
      C++}}{http://www.openwatcom.org} compiler is now supported.
  Updated documentation.
\item[version 1.15] (2003/06/18) 
  The following changes have been contributed by Etienne Lachance:
  \begin{itemize}
  \item Updated documentation.
  \item Definitions in file \texttt{gnugraph.cpp} are now in
    \texttt{gnugraph.h}.
  \item Class \texttt{Plot2d}, \texttt{GNUcurve} are now using
    \textsf{STL} \texttt{string} instead of \texttt{char*}.
  \item Added member functions \texttt{jacobian\_dot()} and
    \texttt{jacobian\_DLS\_inv()} in all robot classes.
  \item Added class \texttt{Config} to read configuration file.
  \item Replaced \texttt{Robot\_basic(const char *filename)} by
    \texttt{Robot\_basic(const string \& filename)}. The new
    constructor uses the class \texttt{Config}.
  \item Provided \texttt{Plot\_file} class to generate graphics from a
    data file.
  \item Added the following \texttt{Quaternion} class member
    functions: \texttt{exponential}, \texttt{logarithm},
    \texttt{dot\_product}, \texttt{dot}, \texttt{E}.
  \item Fixed bugs in \texttt{IO\_matrix\_file} class.
  \item Developed linearized equations for modified DH notations. The
    equations are implemented in \texttt{dq\_torque},
    \texttt{dqp\_torque}, \texttt{dtau\_dq} and \texttt{dtau\_dqp}.
  \item Added examples in \texttt{demo.cpp} related to
    \texttt{IO\_matrix\_file}, \texttt{Plot\_file} and
    \texttt{Config}.
  \end{itemize}
\item[version 1.14] (2003/04/17) Updated documentation. The
  \textsf{Watcom} compiler is no longer supported (problems with
  \textsf{STL} and streams). The following changes have been
  contributed by Etienne Lachance:
  \begin{itemize}
  \item The classes \texttt{RobotMotor} and \texttt{mRobotMotor} no
    longer exist and are now integrated in the \texttt{Robot} and
    \texttt{mRobot} classes. 
  \item The \texttt{Robot} and \texttt{mRobot} classes are now derived
    from the \texttt{Robot\_basic} virtual class.
  \item Removed class \texttt{mlink}. DH and modified DH parameters
    are now included in \texttt{link}.
  \item Added \texttt{kine\_pd()}.
  \item Created a new \texttt{torque} member function that allowed to
    have load on last link.
  \item Fixed bug in modified DH dynamics.
  \item Added a class \texttt{Quaternion}.
  \item Added the program \texttt{rtest} to compare results with Peter
    Corke \htmladdnormallink{\textsf{MATLAB}
      toolbox}{http://www.cat.csiro.au/cmst/staff/pic/robot/}.
  \item Added member function \texttt{set\_plot2d} to generate plots
    using the \texttt{Plot2d} class.
  \item Added utility class \texttt{IO\_matrix\_file} dealing with
    data files (not documented yet).
  \end{itemize}

\item[version 1.13] (2002/08/09) Moved the arrays of
  \texttt{ColumnVector} to the constructors for the dynamics and
  linearized dynamics for a $\approx 10 \%$ gain in speed (thanks to
  Etienne Lachance for the suggestion). Added the \texttt{mRobot} and
  \texttt{mRobotMotor} classes using the modified Denavit-Hartenberg
  notation. Updated documentation.

\item[version 1.12] (2002/02/04) Upgraded the matrix library from
  \textsf{NEWMAT09} to \textsf{NEWMAT10}.

\item[version 1.11] (2001/06/06) Fixed bugs for prismatic joints in
  the dynamics routines (reported by Hassan Abedi).  Updated
  documentation.
  
\item[version 1.10] (2001/04/30) Changed the license to GNU General
  Public License. Workspace for MS \textsf{Visual C++} 6.0. New makefiles using
  implicit rules. New class \texttt{RobotMotor} that includes motors
  parameters (rotor inertia, gear ratio and friction coefficients).
  Updated documentation.

\item[version 1.09] (98/09/27) Makefile for \textsf{MS Visual C++} 6.0.  

\item[version 1.08] (98/06/1) Changes to  {\tt robot.cpp} and {\tt robot.h} to avoid the
warning messages: \par
{\small \verb$initialization of non-const reference `*' from rvalue `*'$} \par
Fixed function {\tt ieulzxz} in {\tt homogen.cpp} thanks to Kilian Pohl. 

\item[version 1.07] (98/05/12) The {\tt bench.cpp} program is more
  portable.  Simpler makefile for\textsf{ Borland C++}. New targets in
  makefiles ({\tt clean} and {\tt veryclean}).  Removed the CVS {\tt
    Log} tags from the sources. Compiler option {\tt -O} now works
  under \textsf{gcc} 2.7.2 thanks to the new {\tt newmat.h} provided
  by Robert Davies.

\item[version 1.06] (97/11/21) The function {\tt inv\_kin} modified to
use the Jacobian by default in the iterative procedure ($\approx 1.8\times$ faster). 
Updated documentation.

\item[version 1.05] (97/11/17) Added make file for\textsf{ GNU G++}
  under \textsf{ Windows 95/NT} using
  \htmladdnormallink{\textsf{Cygnus GNU-Win32}
    compiler}{http://sources.redhat.com/cygwin/}.  Added optimization
  flags under \textsf{GNU G++}. Updated documentation.
  
\item[version 1.04] (97/11/14) Added make file for \textsf{GNU G++}
  and graphic support through \textsf{gnuplot} (2d plots). Updated
  documentation.

\item[version 1.03] (97/11/01) Added adaptive step size integration.
Changes to the documentation.

\item[version 1.02] (97/10/21) Upgraded the matrix library from
  \textsf{NEWMAT08A} to NEWMAT09. New directory structure : {\tt
    newmat08} is replaced by {\tt newmat}. Conditional compilation of
  {\tt delete []} for pre 2.1 C++ compilers has been removed since
  \textsf{NEWMAT09} no longer supports these compilers.  Minor changes
  to the documentation.

\item[version 1.01] (97/01/17) Conditional compilation of {\tt delete []} for pre 2.1 C++ compilers.
Changes to the documentation.


\item[version 1.0] (96/12/15) First public release of the package.
\end{description}
\newpage

\section{Files in the distribution}

{\small
\begin{tabular}{lll}
{\tt readme} & {\tt txt} & readme file \\
{\tt makefile} & {\tt gcc} & make file for GNU G++ Linux \\
{\tt makefile} & {\tt gccOSX} & make file for GNU G++ MAC OS X \\
{\tt makefile} & {\tt gw32} & make file for Cygwin (Win32)  \\
{\tt makefile} & {\tt bc5} & make file for Borland C++ 4.5, 5.x (Win32) \\
{\tt makefile} & {\tt vcpp} & make file for Visual C++ 5.0 and 6.0(Win32) \\
{\tt makefile} & {\tt qnx} & make file for QNX \\
{\tt CMakeLists} & {\tt txt} & Confguration file for CMake \\
{\tt roboop} & {\tt bkl} & Configuration file for Bakefile \\
{\tt roboop} & {\tt dsw} & workspace for Visual C++ 6.0 (Win32) \\
{\tt bench} & {\tt dsp} & project file used by \texttt{roboop.dsw} \\
{\tt demo} & {\tt dsp} & project file used by \texttt{roboop.dsw} \\
{\tt demo\_2dof\_pd} & {\tt dsp} & project file used by \texttt{roboop.dsw} \\
{\tt newmat} & {\tt dsp} & project file used by \texttt{roboop.dsw} \\
{\tt roboop} & {\tt dsp} & project file used by \texttt{roboop.dsw} \\
{\tt rtest} & {\tt dsp} & project file used by \texttt{roboop.dsw} \\
{\tt roboop} & {\tt sln} & solution for Visual C++ 7.0 (Win32) \\
{\tt bench} & {\tt vcproj} & project file used by \texttt{roboop.sln} \\
{\tt demo} & {\tt vcproj} & project file used by \texttt{roboop.sln} \\
{\tt demo\_2dof\_pd} & {\tt vcproj} & project file used by \texttt{roboop.sln} \\
{\tt newmat} & {\tt vcproj} & project file used by \texttt{roboop.sln} \\
{\tt roboop} & {\tt vcproj} & project file used by \texttt{roboop.sln} \\
{\tt rtest} & {\tt vcproj} & project file used by \texttt{roboop.sln} \\
{\tt demo} & {\tt txt} & output of the demo program \\ \\
{\bf newmat} & & directory of the matrix library \textsf{NEWMAT11} \\
& & see the file {\tt nm11.htm} \\ \\
{\bf docs} & & documentation directory \\
{\tt gnugpl} & {\tt txt} & GNU General Public License \\
{\tt gnulgpl} & {\tt txt} & GNU Lesser General Public License \\
{\tt robot} & {\tt ps} & documentation in {\tt postscript} format \\
{\tt robot} & {\tt pdf} & documentation in {\tt PDF} format \\ \\
{\bf doxy} & & \textsf{Doxygen} documentation directory \\
\texttt{roboop\_doxygen} & & \textsf{Doxygen} configuration file
\end{tabular}

\begin{tabular}{lll}
{\bf source} & & the \textsf{ROBOOP} program source directory \\
{\tt CMakeLists} & {\tt txt} & Confguration file for CMake \\
{\tt robot} & {\tt h} & header file \\
{\tt clik} & {\tt h} & header file for {\tt CLIK} \\
{\tt config} & {\tt h} & header file for configuration class\\
{\tt controller} & {\tt h} & header file for controllers\\
{\tt control\_select} & {\tt h} & header file for Control\_Select class\\
{\tt dynamics\_sim} & {\tt h} & header file for Dynamics class\\
{\tt gnugraph} & {\tt h} & header file for the graphics \\
{\tt quaternion} & {\tt h} & header file for the quaternions \\
{\tt stewart} & {\tt h} & header file for the Stewart classs \\
{\tt trajectory} & {\tt h} & header file for the splines \\
{\tt utils} & {\tt h} & header file utility functions \\
{\tt bench   } & {\tt cpp} & benchmark program file \\
{\tt clik   } & {\tt cpp} & closed loop inverse kinematics {\tt CLIK} \\
{\tt comp\_dq } & {\tt cpp} & simplified version of {\tt delta\_t } with no {\tt dqp} and {\tt dqpp} \\
{\tt comp\_dqp} & {\tt cpp} & simplified version of {\tt delta\_t } with no {\tt dq} and {\tt dqpp} \\
{\tt config    } & {\tt cpp} & configuration class members functions \\
{\tt controller} & {\tt cpp} & some controllers functions\\
{\tt control\_select} & {\tt cpp} & controller selection functions\\
{\tt delta\_t } & {\tt cpp} & compute torque variation w/r to {\tt dq}, {\tt dqp} and {\tt dqpp} \\
{\tt demo    } & {\tt cpp} & demo program file \\
{\tt demo\_2dof\_pd} & {\tt cpp} & demo program file \\
{\tt dynamics} & {\tt cpp} & dynamics functions  \\
{\tt dynamics\_sim} & {\tt cpp} & simulation dynamics functions\\
{\tt gnugraph } & {\tt cpp} & graphics functions \\
{\tt homogen } & {\tt cpp} & homogeneous transform functions \\
{\tt impedance} & {\tt cpp} & impedance controller \\
{\tt invkine } & {\tt cpp} & inverse kinematics functions  \\
{\tt kinemat } & {\tt cpp} & kinematics functions  \\
{\tt quaternion } & {\tt cpp} & quaternions functions  \\
{\tt robot   } & {\tt cpp} & constructors and other stuff \\
{\tt rtest   } & {\tt cpp} & testing program file \\
{\tt test   } & {\tt txt} & testing data file \\
{\tt sensitiv} & {\tt cpp} & partial derivatives of robot dynamics \\
{\tt stewart} & {\tt cpp} & implemantation of the Stewart classs \\
{\tt trajectory} & {\tt cpp} & translation and rotation splines \\
{\tt utils} & {\tt cpp} & miscellaneous \\ \\
\end{tabular}

\begin{tabular}{lll}
{\bf conf} & & configuration files directory \\
\texttt{pd\_2dof} & \texttt{conf} & PD controller parameters for the 2 dof robot \\
\texttt{puma560\_dh} & \texttt{conf} & PUMA robot parameters standard D-H \\
\texttt{puma560\_mdh} & \texttt{conf} & PUMA robot parameters modified D-H \\
\texttt{q\_2dod} & \texttt{dat} & desired trajectory for the 2 dof robot \\
\texttt{rhino560\_dh} & \texttt{conf} & RHINO robot parameters standard D-H \\
\texttt{rhino560\_mdh} & \texttt{conf} & RHINO robot parameters modified D-H \\
\texttt{rr\_dh} & \texttt{conf} & 2 dof robot parameters standard D-H \\
\texttt{stewart} & \texttt{conf} & a Stewart platform parameters file 
\end{tabular}
}

\section{Doxygen documentation}

Source code now has \textsf{Doxygen} compatible documentation. To
obtain the documentation (under \textsf{Linux}) simply run
\textsf{doxygen roboop\_doxygen} in the doxy directory. It will
creates \texttt{html} and \texttt{latex} directories.

The main html page can be accessed using the \texttt{index.html} file.
To obtain the latex documentation simply run the \texttt{Makefile} in
the \texttt{latex} directory.

\chapter{Reference manual}

This package uses data types defined by the \textsf{NEWMAT11} matrix
library:
\begin{itemize}
\item {\tt Real} : the type for floating point values. It can be 
either a {\tt float} or a {\tt double}  as defined in the header file 
{\tt include.h} in the {\tt newmat} directory.

\item {\tt Matrix} : the type for matrices as defined in the
    \textsf{NEWMAT11} documentation.

\item {\tt ColumnVector} : a type for column vectors derived from {\tt Matrix}.

\item {\tt ReturnMatrix} : the type used by functions for returning any type of 
matrix ({\tt Matrix, ColumnVector, RowVector,} etc). 
\end{itemize}

The file {\tt demo.cpp} presents examples for the use of some functions in the 
package. The time required to compute some functions for a 6 dof robot can be obtained
with the file {\tt bench.cpp}.
\section{3D homogeneous transforms}

In this section, functions dealing with $4 \times 4$ homogeneous transform
matrices are described.

\newpage

\subsection*{eulzxz}

\subsubsection*{Syntax}
\begin{verbatim}
ReturnMatrix eulzxz(const ColumnVector & a);
\end{verbatim}
\subsubsection*{Description}
Given a column vector {\tt a} 
\begin{eqnarray}
\matr{c}{\gamma_1 \\ \beta \\ \gamma_2}
\end{eqnarray}
this function returns the homogeneous transform matrix given by 
\begin{eqnarray}
\mbold{Rot}(z,\gamma_1) \mbold{Rot}(x,\beta) \mbold{Rot}(z,\gamma_2)
\end{eqnarray}

\noindent {\bf Note: } the column vector {\tt a} must have a length of at least 3. 
Only the first 3 elements are used.

\subsubsection*{Return Value}

{\tt Matrix}

\newpage

\subsection*{ieulzxz}
\subsubsection*{Syntax}
\begin{verbatim}
ReturnMatrix ieulzxz(const Matrix & R);
\end{verbatim}
\subsubsection*{Description}
Given a homogeneous transform matrix {\tt R}, this function returns a column vector
\begin{eqnarray}
\matr{c}{\gamma_1 \\ \beta \\ \gamma_2}
\end{eqnarray}
such that the $3 \times 3$ rotation bloc of the matrix
\begin{eqnarray}
\mbold{Rot}(z,\gamma_1) \mbold{Rot}(x,\beta) \mbold{Rot}(z,\gamma_2)
\end{eqnarray}
is equal to the $3 \times 3$ rotation bloc of the matrix {\tt R}.

%{\bf Note: } 

\subsubsection*{Return Value}

{\tt ColumnVector}.

\newpage

\subsection*{irotk}
\subsubsection*{Syntax}
\begin{verbatim}
ReturnMatrix irotk(const Matrix & R);
\end{verbatim}
\subsubsection*{Description}
Given a homogeneous transform matrix {\tt R}, this function returns a column vector
\begin{eqnarray}
\matr{c}{\mbold{k} \\ \theta}
\end{eqnarray}
with $\mbold{k}$ a unit vector such that the $3 \times 3$ rotation bloc of the matrix
\begin{eqnarray}
\mbold{Rot}(\mbold{k}, \theta)
\end{eqnarray}
is equal to the $3 \times 3$ rotation bloc of the matrix {\tt R}.

%{\bf Note: } 

\subsubsection*{Return Value}

{\tt ColumnVector}.

\newpage

\subsection*{irpy}
\subsubsection*{Syntax}
\begin{verbatim}
ReturnMatrix irpy(const Matrix & R);
\end{verbatim}
\subsubsection*{Description}
Given a homogeneous transform matrix {\tt R}, this function returns a column vector
\begin{eqnarray}
\matr{c}{\alpha \\ \beta \\ \gamma}
\end{eqnarray}
such that the $3 \times 3$ rotation bloc of the matrix
\begin{eqnarray}
\mbold{Rot}(z,\gamma) \mbold{Rot}(y,\beta) \mbold{Rot}(x,\alpha)
\end{eqnarray}
is equal to the $3 \times 3$ rotation bloc of the matrix {\tt R}.

%{\bf Note: } 

\subsubsection*{Return Value}

{\tt ColumnVector}.

\newpage

\subsection*{rotd}
\subsubsection*{Syntax}
\begin{verbatim}
ReturnMatrix rotd(const Real theta, 
                  const ColumnVector & k1, 
                  const ColumnVector & k2);
\end{verbatim}
\subsubsection*{Description}
This function returns the matrix of a rotation of an angle {\tt theta} around 
the oriented line segment defined by the points {\tt k1} and {\tt k2}.

\noindent {\bf Note: } the column vectors {\tt k1} and {\tt k2} must have a length of at least 3. 
Only the first 3 elements are used.

\subsubsection*{Return Value}

{\tt Matrix}

\newpage

\subsection*{rotk}

\subsubsection*{Syntax}
\begin{verbatim}
ReturnMatrix rotk(const Real theta, 
                  const ColumnVector & k);
\end{verbatim}
\subsubsection*{Description}
This function returns the matrix of a rotation of an angle {\tt theta} around 
the vector {\tt k}.
\begin{eqnarray}
\mbold{Rot}(\mbold{k}, \theta)
\end{eqnarray}

\noindent {\bf Note: } the column vector {\tt k} must have a length of at least 3. 
Only the first 3 elements are used.

\subsubsection*{Return Value}

{\tt Matrix}

\newpage

\subsection*{rpy}
\subsubsection*{Syntax}
\begin{verbatim}
ReturnMatrix rpy(const ColumnVector & a);
\end{verbatim}
\subsubsection*{Description}
Given a column vector {\tt a} 
\begin{eqnarray}
\matr{c}{\alpha \\ \beta \\ \gamma}
\end{eqnarray}
this function returns the homogeneous transform matrix given by 
\begin{eqnarray}
\mbold{Rot}(z,\gamma) \mbold{Rot}(y,\beta) \mbold{Rot}(x,\alpha)
\end{eqnarray}

\noindent {\bf Note: } the column vector {\tt a} must have a length of at least 3. 
Only the first 3 elements are used.

\subsubsection*{Return Value}

{\tt Matrix}

\newpage

\subsection*{rotx, roty, rotz}

\subsubsection*{Syntax}
\begin{verbatim}
ReturnMatrix rotx(const Real alpha);
ReturnMatrix roty(const Real beta);
ReturnMatrix rotz(const Real gamma);
\end{verbatim}
\subsubsection*{Description}
These functions return the elementary rotation matrices:
\begin{eqnarray}
\mbold{Rot}(x,\alpha) & = & 
\matr{cccc}{
1 & 0 & 0 & 0 \\
0 & \cos \alpha & - \sin \alpha & 0 \\
0 & \sin \alpha & \cos \alpha & 0 \\
0 & 0 & 0 & 1
} \\
\mbold{Rot}(y,\beta) & = & 
\matr{cccc}{
\cos \beta & 0 & \sin \beta & 0\\
0 & 1 & 0 & 0 \\
- \sin \beta & 0 & \cos \beta & 0 \\
0 & 0 & 0 & 1
} \\
\mbold{Rot}(z,\gamma) & = & 
\matr{cccc}{
\cos \gamma & - \sin \gamma & 0 & 0 \\
\sin \gamma & \cos \gamma & 0 & 0 \\
0 & 0 & 1 & 0 \\
0 & 0 & 0 & 1
}
\end{eqnarray}

%{\bf Note: } 

\subsubsection*{Return Value}

{\tt Matrix}

\newpage

\subsection*{trans}

\subsubsection*{Syntax}
\begin{verbatim}
ReturnMatrix trans(const ColumnVector & a);
\end{verbatim}
\subsubsection{Description}
Given a column vector {\tt a}, this function returns the following matrix:
\begin{eqnarray}
\mbold{Trans}(a) & = & 
\matr{cccc}{
1 & 0 & 0 & a_1 \\
0 & 1 & 0 & a_2 \\
0 & 0 & 1 & a_3 \\
0 & 0 & 0 & 1
}
\end{eqnarray}

\noindent {\bf Note: } the column vector {\tt a} must have a length of at least 3. 
Only the first 3 elements are used.

\subsubsection*{Return Value}

{\tt Matrix}

\newpage

\section{The \texttt{Quaternion} class}

The \texttt{Quaternion} class deals with quaternions. Unit quaternions
are used to represent rotations. It is composed of two elements: a
scalar $s$ (\texttt{Real s\_}) and a vector $\mbold{v}$
(\texttt{ColumnVector v\_}) representing a quaternion
(see\cite{Chou92}).
\begin{eqnarray}
 q & = & w + xi + yj + zk \\
   & = & (s, v) 
\end{eqnarray}

An object of this class can be initialize with no parameter ($s=1$ and
$\mbold{v=0}$), from an other unit quaternion, from an angle of
rotation around a unit vector, from a rotation matrix, from a
quaternion object or from the four components of a quaternion. The
constructors does not guarantee that quaternions will be unit.

\subsection*{constructors}

\subsubsection*{Syntax}
\begin{verbatim}
Quaternion();
Quaternion(const Quaternion & q);
Quaternion(const Real angle_in_rad, const ColumnVector & axis);
Quaternion(const Real s, const Real v1, const Real v2, const Real v3);
Quaternion(const Matrix & R);
Quaternion & operator=(const Quaternion & q);
\end{verbatim}
\subsubsection{Description}
\texttt{Quaternion} object constructors, copy constructor and equal operator.

\subsubsection*{Return Value}

None

\newpage

\subsection*{operators}

\subsubsection*{Syntax}
\begin{verbatim}
Quaternion   operator+(const Quaternion & q)const;
Quaternion   operator-(const Quaternion & q)const;
Quaternion   operator*(const Quaternion & q)const;
Quaternion   operator*(const ColumnVector & vec)const;
Quaternion   operator*(constReal c)const;
Quaternion   operator/(const Quaternion & q)const;
Quaternion   operator/(constReal c)const;
\end{verbatim}
\subsubsection{Description}
The operators $+$, $-$, $*$ and $/$ for quaternion are
implemented.
The operators $*$ and $/$ will generate unit quaternions 
only if the quaternions involve are unity.

\subsubsection*{Return Value}

{\tt Quaternion}

\newpage

\subsection*{conjugate and inverse}

\subsubsection*{Syntax}
\begin{verbatim}
Quaternion   conjugate()const;
Quaternion   i()const;
\end{verbatim}
\subsubsection{Description}
Compute the conjugate of the quaternion (or the inverse if it's a
unit quaternion). The conjugate is defined as
\begin{eqnarray}
 q^{*} & = & w - xi - yj - zk \\
       & = & (s,-v)
\end{eqnarray}

\subsubsection*{Return Value}

{\tt Quaternion}

\newpage

\subsection*{exponential and logarithm}

\subsubsection*{Syntax}
\begin{verbatim}
Quaternion   exp()const;
Quaternion   Log()const;
Quaternion   power(const Real t)const;
\end{verbatim}
\subsubsection{Description}
A unit quaternion can be represented by $q = cos(\theta) + u
sin(\theta)$. Euler's identity for complex numbers generalizes to
quaternions $exp(u\theta) = cos(\theta) + u sin(\theta)$, where
$exp(x)$ is replace by $exp(u\theta)$ and $uu$ is replace by $-1$.
With this identity we obtain the exponential of the quaternion $q =
(0,\theta v)$, where $q$ is not necessary a unit quaternion. It is
then possible to define the logarithm and the power of a unit
quaternion \cite{Dam98}.
\begin{eqnarray}
 Log(q) &=& Log\big(\cos(\theta) + u \sin(\theta)\big) =
 Log\big(exp(u\theta)\big) = u\theta \\
 q^t &=& \cos(t\theta) + u \sin(t\theta) 
\end{eqnarray}
$Log(q)$ is not necessary a unit quaternion even if $q$ is a unit
quaternion.

\subsubsection*{Return Value}

{\tt Quaternion} for \texttt{exp, Log}

\newpage

\subsection*{dot\_product}

\subsubsection*{Syntax}
\begin{verbatim}
Real dot_prod(const Quaternion & q)const;
\end{verbatim}
\subsubsection{Description}
Compute the dot product of quaternions.

\subsubsection*{Return Value}

{\tt Real}

\newpage

\subsection*{quaternion time derivative}

\subsubsection*{Syntax}
\begin{verbatim}
Quaternion   dot(const ColumnVector & w, const short sign)const;
ReturnMatrix E(const short sign)const; 
\end{verbatim}
\subsubsection{Description}
The quaternion time derivative is obtain from the quaternion
propagation law \cite{Dam98}.
\begin{eqnarray}
 \dot{s} &=& -\frac{1}{2}v^Tw \\
 \dot{v} &=& \frac{1}{2}E(s,v)w 
\end{eqnarray}
where
\begin{eqnarray}
 \begin{array}{cc}
   E = \eta I - S(\epsilon) & \textrm{in base frame} \\
   E = \eta I + S(\epsilon) & \textrm{in body frame}
  \end{array}
\end{eqnarray}

\noindent
The choice of reference system (base or body) for \mbold{w} is assign
by \mbold{sign}. A value of $1$ is for base frame while $-1$ is for
body frame.

\subsubsection*{Return Value}

{\tt Quaternion} for \texttt{dot}

\noindent \texttt{Matrix} for \texttt{E}

\newpage

\subsection*{unit and norm}

\subsubsection*{Syntax}
\begin{verbatim}
Quaternion & unit();
Real norm()const;
\end{verbatim}
\subsubsection{Description}
\texttt{unit()} makes the quaternion a unit quaternion,
\texttt{norm()} computes and returns the norm of the quaternion.
\texttt{norm\_sqr()} computes and returns the square norm of the quaternion.

\subsubsection*{Return Value}

{\tt Quaternion} for \texttt{unit()}

\noindent \texttt{Real} for \texttt{norm()} and \texttt{norm\_sqr()}

\newpage

\subsection*{s and v}

\subsubsection*{Syntax}
\begin{verbatim}
Real s()const;
void set_s(const Real s);
ReturnMatrix v()const;
void set_v(const ColumnVector & v);
\end{verbatim}
\subsubsection{Description}
The functions \texttt{s()} and \texttt{v()} returns one of the
components of a quaternion ($s$ or $\mbold{v}$), while
\texttt{set\_s()} and \texttt{set\_v()} can assign a value to one of
the components.

\subsubsection*{Return Value}

None for \texttt{set\_s()} and \texttt{set\_v()}

\noindent {\tt Real} for \texttt{s()}

\noindent \texttt{Matrix} for \texttt{v()}

\newpage

\subsection*{Rotation matrices}

\subsubsection*{Syntax}
\begin{verbatim}
ReturnMatrix R() const;
ReturnMatrix T() const;
\end{verbatim}
\subsubsection{Description}
Returns a rotation matrix from the quaternion (\texttt{R()} returns a
$3 \times 3$ matrix and \texttt{T()} returns a $4 \times 4$ matrix).

\subsubsection*{Return Value}

{\tt Matrix}

\newpage

\subsection*{Omega, $\omega$}

\subsubsection*{Syntax}
\begin{verbatim}
ReturnMatrix Omega(const Quaternion & q, const Quaternion & q_dot);
\end{verbatim}
\subsubsection{Description}
Omega is not a member function of the class \texttt{Quaternion}. The
function returned the angular velocity obtain from a quaternion and
it's time derivative. Like the member function \texttt{dot}, it use
the quaternions propagation law \cite{Dam98}.

\subsubsection*{Return Value}

{\tt ColumnVector}

\newpage

\subsection*{Slerp}

\subsubsection*{Syntax}
\begin{verbatim}
Quaternion Slerp(const Quaternion & q0, const Quaternion & q1, 
                 const Real t);
\end{verbatim}
\subsubsection{Description}
Slerp stands for Spherical Linear Interpolation.  Slerp is not a
member function of the class Quaternion.  The quaternions $q_0$ and
$q_1$ needs to be unit quaternions.  It returns a unit quaternion. As
the parameter $t$ uniformly varies between 0 and 1, the values $q(t)$
are required to uniformly vary along the circular arc from $q_0$ to
$q_1$.

It is customary to choose the sign $G$ on $q_1$ so that $q_0 \cdot
Gq_1 \geq 0$ (the angle between $q_0$ and $Gq_1$ is acute). This
choice avoids extra spinning caused by the interpolated rotations
\cite{Dam98}. For unit quaternions Slerp is defined as
\begin{eqnarray}
 q = \Bigg\{
  \begin{array}{cc}
    q_0(q_0^{-1}q_1)^t & \textrm{if $q_0 \cdot q_1 \geq 0$} \\
    q_0(q_0^{-1}(-q_1))^t & \textrm{otherwise} 
  \end{array}
\end{eqnarray}


\subsubsection*{Return Value}

Quaternion

\newpage

\subsection*{Slerp\_prime}

\subsubsection*{Syntax}
\begin{verbatim}
Quaternion Slerp_prime(const Quaternion & q0, const Quaternion & q1, 
                       const Real t);
\end{verbatim}
\subsubsection{Description}
Slerp\_prime represent the Slerp derivative.  Slerp\_prime is not a
member function of the class Quaternion.  The quaternions $q_0$ and
$q_1$ needs to be unit quaternions.  It does not necessary returns a
unit quaternion.

It is customary to choose the sign $G$ on $q_1$ so that $q_0 \cdot
Gq_1 \geq 0$ (the angle between $q_0$ and $Gq_1$ is acute). This
choice avoids extra spinning caused by the interpolated rotations
\cite{Dam98}. For unit quaternions Slerp is defined as
\begin{eqnarray}
 q = \Bigg\{
  \begin{array}{cc}
    Slerp(q_0, q_1, t)Log(q_0^{-1}q_1) & \textrm{if $q_0 \cdot q_1 \geq 0$} \\
    Slerp(q_0, q_1, t)Log(q_0^{-1}(-q_1)) & \textrm{otherwise} \\
  \end{array}
\end{eqnarray}

\subsubsection*{Return Value}

Quaternion

\newpage

\subsection*{Squad}

\subsubsection*{Syntax}
\begin{verbatim}
Quaternion Squad(const Quaternion & p, const Quaternion & a, 
                 const Quaternion & b, const Quaternion & r, 
                 const Real t);
\end{verbatim}
\subsubsection{Description}
Squad stands for Spherical Cubic Interpolation. Squad is not a member
function of the class Quaternion. The quaternions $p$, $a$, $b$ and
$r$ needs to be unit quaternions. It returns a unit quaternion.

Squad uses an iterative of three slerps. Suppose four quaternions,
$p$, $a$, $b$ and $r$ as the ordered vertices of quadrilateral.
Interpolate $c$ along $p$ to $q$ using slerp and $d$ along $a$ to $b$
also using slerp. Now interpolate $q$ along $c$ to $d$ \cite{Dam98}.
Squad is defined as
\begin{eqnarray}
  q = Slerp(Slerp(p,r,t),Slerp(a,b,t),2t(1-t));
\end{eqnarray}
 
\subsubsection*{Return Value}

Quaternion

\newpage

\subsection*{Squad\_prime}

\subsubsection*{Syntax}
\begin{verbatim}
Quaternion Squad_prime(const Quaternion & p, const Quaternion & a, 
                       const Quaternion & b, const Quaternion & q, 
                       const Real t);
\end{verbatim}
\subsubsection{Description}
Squad\_prime represent the Squad derivative.  Squad\_prime is not a
member function of the class Quaternion.

\subsubsection*{Return Value}

Quaternion

\newpage

\section{The \texttt{Robot} and \texttt{mRobot} classes}

The \texttt{Robot} and \texttt{mRobot} classes are composed of the
following data elements:
\begin{itemize}
\item the number of degree of freedom $n$ ({\tt int dof});
\item the gravity acceleration vector ($-\mbold{g}$) expressed in the
  base frame \\({\tt ColumnVector gravity});
\item one array of dimension $n$ of \texttt{Link} object elements
  ({\tt Link *links});
\end{itemize}
and the member functions providing the different algorithms implementation 
(see tables~\ref{tab:commandsum1}--\ref{tab:commandsum3}).

The \texttt{Link} class (see table~\ref{tab:link}) encapsulates all
the data and functionality required to characterize a single ``link''
as it is defined by Denavit and Hartenberg (standard notation
\cite{Denavit55}, or modified notation \cite{craig}). It is initialized
by providing the joint type ({\tt int joint\_type}: revolute=0,
prismatic=1) and the parameters $\theta$, $d$, $a$, $\alpha$ ({\tt
  Real theta, d, a, alpha}) and a boolean value \texttt{Bool DH}
(\texttt{true}=standard \texttt{false}=modified) It also contains the
inertial parameters data: mass $m$ ({\tt Real m}), center of mass
position vector $\mbold{r}$ ({\tt ColumnVector r}) and inertia tensor
matrix $\mbold{I}_c$ ({\tt Matrix I}). In this case, $\mbold{r}$ is
given with respect to the link coordinate frame and $\mbold{I}_{c}$ is
with respect to a coordinate frame parallel to the link coordinate
frame and located at the center of mass of $m$.  The dynamic model
takes into account the motors inertia, gear ratio and frictions. The
values \texttt{Im} and \texttt{Gr} representing respectively the
motors rotor inertia $I_m$ and gear ratio $G_r$; \texttt{B} and
\texttt{Cf} representing respectively the motors viscous $B$ and
Coulomb friction $C_f$ coefficients:
  \begin{eqnarray*}
    \tau_{f}& = &  B \dot{q} + C_f \mbox{sign}(\dot{q})
  \end{eqnarray*}

\begin{table}[htbp]
  \caption{The \texttt{Link} class data parameters}
  \label{tab:link}
  \begin{center}
    \begin{tabular}{||ll|ll||ll||}
      \hline
      \hline
      \multicolumn{2}{||l|}{Kinematic} & \multicolumn{2}{l||}{Inertial} 
      & \multicolumn{2}{l||}{Motor} \\
      \hline
      \hline
      {\tt int} & {\tt joint\_type} & {\tt Real} & {\tt m} 
      & \texttt{Real} & \texttt{Im} \\
      {\tt Real} & {\tt theta, d, a, alpha} & {\tt ColumnVector} & {\tt r}
      & \texttt{Real} & \texttt{Gr} \\
      {\tt Real} & {\tt joint\_offset} & {\tt Matrix} & {\tt I}
      & \texttt{Real} & \texttt{B} \\
      {\tt ColumnVector} & {\tt p} & & & \texttt{Real} & \texttt{Cf} \\
      {\tt Matrix} & {\tt R,} & & & & \\
      \texttt{Bool} & \texttt{DH} & & & & \\
      \texttt{Real} & \texttt{theta\_min}, \texttt{theta\_max} & & & & \\
      \texttt{Real} & \texttt{joint\_offset} & & & & \\
      \hline
      \hline
    \end{tabular}
  \end{center}
\end{table}

On initialization, the constructor sets up the matrices $\mbold{R}$ and $\mbold{p}$ such that
\begin{eqnarray}
\mbold{R} & = & \matr{ccc}{ 
\cos \theta & -\cos \alpha \sin \theta & \sin \alpha \sin \theta \\
\sin \theta & \cos \alpha \cos \theta & -\sin \alpha \cos \theta \\
0 & \sin \alpha & \cos \alpha } \\
\mbold{p} & = & \matr{c}{a \cos \theta \\ a \sin \theta \\ d}
\end{eqnarray}
for the standard D-H notation and
\begin{eqnarray}
  \mbold{R} & = & \matr{ccc}{ 
    \cos \theta & -\sin \theta & 0 \\
    \cos \alpha \sin \theta & \cos \alpha \cos \theta & -\sin \alpha \\
    \sin \alpha \sin \theta & \sin \alpha \cos \theta & \cos \alpha } \\
  \mbold{p} & = & \matr{c}{a \\ - d \sin \alpha \\ d \cos \alpha }
\end{eqnarray}
for the modified D-H notation.

If the link corresponds to a revolute (prismatic) joint, then
only $\theta$ ($d$) can be changed after the link definition. This is done through 
the member function {\tt transform} which sets the new value of $q$ ($\theta$ or $d$) and
updates the matrices $\mbold{R}$ and $\mbold{p}$ which compose the link homogeneous transform:
\begin{eqnarray}
\mbold{T} & = & \matr{cc}{
\mbold{R} & \mbold{p} \\
0 & 1 }
\end{eqnarray}
Only the changing elements are computed since the data of an instance
of a class is persistent throughout the scope of definition of the
instance (see \cite{eckel}). In standard notation, the elements (3,2)
and (3,3) of $\mbold{R}$ provide storage for $\cos \alpha$ and $\sin
\alpha$ which are computed only once. In modified notation, the
elements (3,3) and (2,3) of $\mbold{R}$ provide storage for $\cos
\alpha$ and $\sin \alpha$. So as to make the implementation faster,
only the elements of $\mbold{R}$ and $\mbold{p}$ involving $\theta$
($d$) are updated with a revolute (prismatic) joint.


\subsection{\texttt{Robot} and \texttt{mRobot} object initialization}

The \texttt{Robot} and \texttt{mRobot} classes provide a default
constructor that creates a 1 dof robot.  A $n_{dof} \times 19$ matrix
containing the kinematic and inertial parameters (as for the
\texttt{Robot} class) can be supplied upon initialization. A $n_{dof}
\times 19$ matrix containing the kinematic and inertial parameters (as
for the \texttt{Robot} class) can be supplied along with a $n_{dof}
\times 4$ matrix providing the motors inertia, gear ratio and friction
coefficients. A $n_{dof} \times 23$ matrix (kinematic, inertial and
motor parameters) can also be used.  The structure of the
initialization matrix is:

\vspace{0.2in}

\begin{tabular}{lll}
Column & Variable & Description \\
\\
1 & $\sigma$ & joint type (revolute=0, prismatic=1) \\
2 & $\theta$ & Denavit-Hartenberg parameter \\
3 & $d$ & Denavit-Hartenberg parameter \\
4 & $a$ & Denavit-Hartenberg parameter \\
5 & $\alpha$ & Denavit-Hartenberg parameter \\
6 & $\theta_{min}$ & minimum value of joint variable \\
7 & $\theta_{max}$ & maximum value of joint variable \\
8 & $\theta_{off}$ & joint offset \\
9 & $m$ & mass of the link \\
10 & $c_x$ & center of mass along axis $x$ \\
11 & $c_y$ & center of mass along axis $y$ \\
12 & $c_z$ & center of mass along axis $z$ \\
13 & $I_{xx}$ & element $xx$ of the inertia tensor matrix \\
14 & $I_{xy}$ & element $xy$ of the inertia tensor matrix \\
15 & $I_{xz}$ & element $xz$ of the inertia tensor matrix \\
16 & $I_{yy}$ & element $yy$ of the inertia tensor matrix \\
17 & $I_{yz}$ & element $yz$ of the inertia tensor matrix \\
18 & $I_{zz}$ & element $zz$ of the inertia tensor matrix \\
19 & $I_m$ & motor rotor inertia \\
20 & $Gr$ & motor gear ratio \\
21 & $B$ & motor viscous friction coefficient \\
22 & $C_f$ & motor Coulomb friction coefficient \\
23 & \texttt{immobile} & flag for the kinematics and inverse kinematics \\
& & (if true joint is locked, if false joint is free)
\end{tabular}

\newpage

\subsection*{constructors}
\subsubsection*{Syntax}

Standard notation:
\begin{verbatim}
Robot(const int ndof=1);
Robot(const Matrix & initrobot);
Robot(const Matrix & initrobot, const Matrix & initmotor);
Robot(const Robot & x);
Robot(const string & filename, const string & robotName);
\end{verbatim}
Modified notation:
\begin{verbatim}
mRobot(const int ndof=1);
mRobot(const Matrix & initrobot_motor);
mRobot(const Matrix & initrobot, const Matrix & initmotor);
mRobot(const mRobot & x);
mRobot(const string & filename, const string & robotName);
\end{verbatim}
\subsubsection*{Description}

\texttt{Robot} and \texttt{mRobot} object constructors, copy
constructor and equal operator.

%{\bf Note: } 

\subsubsection*{Return Value}

None

\newpage

\subsection*{get\_q, get\_qp, get\_qpp}
\subsubsection*{Syntax}
\begin{verbatim}
ReturnMatrix get_q(void);
Real get_q(const int i);
ReturnMatrix get_qp(void);
Real get_qp(const int i);
ReturnMatrix get_qp(void);
Real get_qp(const int i);
\end{verbatim}
\subsubsection*{Description}
These functions return a column vector containing the joint variables
(\texttt{get\_q}), velocities (\texttt{get\_qp}) or accelerations
(\texttt{get\_qpp}) when called with no argument. It returns the
scalar value for the $i^{th}$ joint variable when called with an
integer argument.
%{\bf Note: } 

\subsubsection*{Return Value}

{\tt ColumnVector} or {\tt Real}

\newpage

\subsection*{set\_q, set\_qp, set\_qpp}
\subsubsection*{Syntax}
\begin{verbatim}
void set_q(const ColumnVector & q);
void set_q(const Matrix & q);
void set_q(const Real q, const int i);
void set_qp(const ColumnVector & qp);
void set_qp(const Matrix & qp);
void set_qp(const Real qp, const int i);
void set_qpp(const ColumnVector & qpp);
void set_qpp(const Matrix & qpp);
void set_qpp(const Real qpp, const int i);
\end{verbatim}
\subsubsection*{Description}
These functions set the joint variables (velocities or accelerations)
or the $i^{th}$ joint variable (velocity or acceleration) to {\tt q}
(\texttt{qp} or \texttt{qpp}).
%{\bf Note: } 

\subsubsection*{Return Value}

None

\newpage

\subsection{Kinematics}

The forward kinematic model defines the relation:
\begin{eqnarray}
{}^0 \mbold{T}_n & = & \mbold{G(q)}
\end{eqnarray}
where ${}^0 \mbold{T}_n$ is the homogeneous transform representing the position
and orientation of the manipulator tool (frame $n$) in the base frame $0$.
The inverse kinematic model is defined by
\begin{eqnarray}
\mbold{q} & = & \mbold{G}^{-1}({}^0 \mbold{T}_n)
\end{eqnarray}
In general, this equation allows multiple solutions.

\newpage

\subsection*{inv\_kin}
\subsubsection*{Syntax}
\begin{verbatim}
ReturnMatrix inv_kin(const Matrix & Tobj, const int mj = 0);
ReturnMatrix inv_kin(const Matrix & Tobj, const int mj, 
                     const int endlink, bool & converge);
\end{verbatim}
\subsubsection*{Description}
The inverse kinematic model is computed using a Newton-Raphson technique. 
If {\tt mj == 0}, it is based on the following \cite{Gorla84}:
\begin{eqnarray}
{}^0 \mbold{T}_n (\mbold{q}^*) & = & {}^0 \mbold{T}_n (\mbold{q} + \delta \mbold{q})
\approx {}^0 \mbold{T}_n (\mbold{q}) \delta \mbold{T} ( \delta \mbold{q}) =  \mbold{T}_{obj} \\
\delta \mbold{T} ( \delta \mbold{q}) & = & ({}^0 \mbold{T}_n (\mbold{q}))^{-1} \mbold{T}_{obj} = 
\mbold{I + \Delta} \\
\mbold{\Delta} & = & \matr{cccc}{
0 & -\delta_z & \delta_y & d_x \\
\delta_z & 0 & -\delta_x & d_y \\
-\delta_y & \delta_x & 0 & d_z \\
0 & 0 & 0 & 0} \\
{}^n \delta \mbold{\chi} & = & \matr{cccccc}{
d_x & d_y & d_z & \delta_x & \delta_y & \delta_z 
}^T \\
{}^n \delta \mbold{\chi} & \approx & {}^n \mbold{J}(\mbold{q}) \delta \mbold{q}
\end{eqnarray}
If {\tt mj == 1}, it is based on 
the following Taylor expansion \cite{Gorla84,Uicker67}:
\begin{eqnarray}
{}^0 \mbold{T}_n (\mbold{q}^*) & = {}^0 \mbold{T}_n (\mbold{q} + \delta \mbold{q}) \approx & 
{}^0 \mbold{T}_n (\mbold{q}) + \sum_{i=1}^{n} \dfdx{{}^0 \mbold{T}_n}{q_i} \delta q_i
\end{eqnarray}
The function {\tt dTdqi} computes these partial derivatives.

Given the desired position represented by the homogeneous transform
{\tt Tobj}, this function return the column vector of joint variables
that is corresponding to this position. On return, the value
\texttt{converge} is true if the procedure has converge to values that
give the correct position and false otherwise.

\noindent {\bf Note: } {\tt mj == 0} is faster ($\approx 1.8\times$) than {\tt mj == 1}. 
Also, {\tt mj == 1} might converge when {\tt mj == 0} does not.

\subsubsection*{Return Value}

{\tt ColumnVector}

\newpage

\subsection*{inv\_kin\_rhino}
\subsubsection*{Syntax}
\begin{verbatim}
ReturnMatrix inv_kin_rhino(const Matrix & Tobj, 
                           bool & converge)
\end{verbatim}
\subsubsection*{Description}
This function performs the Rhino robot inverse kinematics.

\subsubsection*{Return Value}

{\tt ColumnVector}

\newpage

\subsection*{inv\_kin\_puma}
\subsubsection*{Syntax}
\begin{verbatim}
ReturnMatrix inv_kin_puma(const Matrix & Tobj, 
                          bool & converge)
\end{verbatim}
\subsubsection*{Description}
This function performs the Puma robot inverse kinematics.

\subsubsection*{Return Value}

{\tt ColumnVector}

\newpage

\subsection*{jacobian}
\subsubsection*{Syntax}
\begin{verbatim}
ReturnMatrix jacobian(const int ref=0);
ReturnMatrix jacobian(const int endlink, const int ref)const;
\end{verbatim}
\subsubsection*{Description}
The manipulator Jacobian defines the relation between the velocities in joint space 
$\dot{\mbold{q}}$ and in the Cartesian space $\dot{\mbold{\chi}}$ expressed in frame $i$:
\begin{eqnarray}
{}^i \dot{\mbold{\chi}} & = & {}^i \mbold{J}(\mbold{q}) \dot{\mbold{q}}
\end{eqnarray}
or the relation between small variations in joint space 
$\delta \mbold{q}$ and small displacements in the Cartesian space $\delta \mbold{\chi}$:
\begin{eqnarray}
{}^i \delta \mbold{\chi} & \approx & {}^i \mbold{J}(\mbold{q}) \delta \mbold{q}
\end{eqnarray}
The manipulation Jacobian expressed in the base frame is given by 
(see \cite{Fu87})
\begin{eqnarray}
{}^0 \mbold{J}(\mbold{q}) & = & \matr{cccc}{
{}^0 \mbold{J}_1(\mbold{q}) & {}^0 \mbold{J}_2(\mbold{q}) & \cdots & {}^0 \mbold{J}_n(\mbold{q})}
\end{eqnarray}
with
\begin{eqnarray}
  {}^0 \mbold{J}_i(\mbold{q}) & = & 
  \matr{c}{ \mbold{z}_{i-1} \times {}^{i-1} \mbold{p}_n \\ \mbold{z}_{i-1}} 
  \mbox{ for a revolute joint} \\
  {}^0 \mbold{J}_i(\mbold{q}) & = & 
  \matr{c}{ \mbold{z}_{i-1} \\ 0} \mbox{ for a prismatic joint} 
\end{eqnarray}
where $\mbold{z}_{i-1}$ and ${}^{i-1} \mbold{p}_n$ are expressed in
the base frame and $\times$ is the vector cross product.  Expressed in
the $i^{th}$ frame, the Jacobian is given by
\begin{eqnarray}
{}^i \mbold{J}(\mbold{q}) & = &
\matr{cc}{
({}^0 \mbold{R}_i)^T & 0 \\ 0 & ({}^0 \mbold{R}_i)^T 
} {}^0 \mbold{J}(\mbold{q})
\end{eqnarray}

This function returns ${}^i \mbold{J}(\mbold{q})$ ($i=0$ when not
specified) for the \texttt{endlink} (last link when not specified).


%{\bf Note: } 

\subsubsection*{Return Value}

{\tt Matrix}

\newpage


\subsection*{jacobian\_dot}
\subsubsection*{Syntax}
\begin{verbatim}
ReturnMatrix jacobian_dot(const int ref=0);
\end{verbatim}
\subsubsection*{Description}   
The manipulator Jacobian time derivative can be used to compute the end effector acceleration due to joints velocities \cite{Angeles97}:
\begin{eqnarray}
  ^i\ddot{\mbold{x}}\ =\  ^i\dot{\mbold{J}}(\mbold{q},\dot{\mbold{q}})\dot{\mbold{q}}
\end{eqnarray}
The Jacobian time derivative expressed in the base frame is given by \cite{Angeles97}
\begin{eqnarray}
  ^0\dot{\mbold{J}}(\mbold{q},\dot{\mbold{q}})\ =\ 
  \left[
    \begin{array}{cccc}
      ^0\dot{\mbold{J}}_1(\mbold{q},\dot{\mbold{q}}) & ^0\dot{\mbold{J}}_2(q,\dot{\mbold{q}}) 
      & \cdots & ^0\dot{\mbold{J}}_n(\mbold{q},\dot{\mbold{q}})
    \end{array}
  \right]
\end{eqnarray}
with
\begin{eqnarray}
  ^0\dot{\mbold{J}}_i(\mbold{q},\dot{\mbold{q}}) &=&
  \begin{array}{cc}
    \left[
      \begin{array}{c}
        \mbold{\omega}_{i-1}\times \mbold{z}_i \\
        \mbold{\omega}_{i-1}\times ^{i-1}\mbold{p}_n + \mbold{z}_i\times ^{i-1}\dot{\mbold{p}}_n
      \end{array}
    \right]  & \textrm{for a revolute joint}
  \end{array} \\
  ^0\dot{\mbold{J}}_i(\mbold{q},\dot{\mbold{q}}) &=&
  \begin{array}{cc}
    \left[
      \begin{array}{c}
        0 \\
        0
      \end{array}
    \right]  & \textrm{for a prismatic joint}
  \end{array}
\end{eqnarray}

where $\mbold{z}_i$ and $^{i-1}\mbold{p}_n$ are expressed in the base
frame and $\times$ is the vector cross product. Expressed in the
$i^{th}$ frame, the Jacobian time derivative is given by

\begin{eqnarray}
  ^i\dot{\mbold{J}}(\mbold{q},\dot{\mbold{q}}) =
  \left[
    \begin{array}{cc}
      (^0\mbold{R}_i)^T & 0 \\
      0 & (^0\mbold{R}_i)^T
    \end{array}
  \right] {^0}\dot{\mbold{J}}(\mbold{q},\dot{\mbold{q}})
\end{eqnarray}

This function returns $^i\dot{\mbold{J}}(\mbold{q},\dot{\mbold{q}})$(i=0 when not specified).

\subsubsection*{Return Value}

{\tt Matrix}
 \newpage 


\subsection*{jacobian\_DLS\_inv}
\subsubsection*{Syntax}
\begin{verbatim}
ReturnMatrix jacobian_DLS_inv(const Real eps, const Real lambda_max, 
                              const int ref=0);
\end{verbatim}

\subsubsection*{Description}   
This function returns the inverse Jacobian Matrix for 6 dof manipulator based on the Damped Least-Squares 
scheme \cite{Chiaverini94}. Using the singular value decomposition, the Jacobian matrix is

\begin{eqnarray}
  J = \sum_{i=1}^6\sigma_i u_i v_i^T
\end{eqnarray}
where $v_i$ and $u_i$ are the input and output vectors, and $\sigma_i$ are the singular values ordered so that
$\sigma_i \geq \sigma_2 \geq \cdots \sigma_r \geq 0$, with $r$ being the rank of $J$. Based on the Damped Least-Squares
the inverse Jacobian can be written as
\begin{eqnarray}
  J^{-1} = \sum_{i=1}^6\frac{\sigma_i}{\sigma_i^2 + \lambda^2}v_iu_i^T
\end{eqnarray}
where $\lambda$ is the damping factor. A singular region can be selected on the basis of the smallest singular value of J.
Outside the region the exact solution is returned, while inside the region a configuration-varying damping factor is 
introduced to obtain the desired approximate solution. This region is defined as
\begin{eqnarray}
  \lambda^2\ =\ \Bigg\{
  \begin{array}{cc}
    0 &     \textrm{if $\sigma_6 \geq \epsilon$} \\
    \Big(1\ -\ (\frac{\sigma_6}{\epsilon})^2 \Big)\lambda^2_{max} & \textrm{otherwise}
  \end{array}
\end{eqnarray}

\subsubsection*{Return Value}

{\tt Matrix}
 \newpage 


\subsection*{kine}
\subsubsection*{Syntax}
\begin{verbatim}
void kine(Matrix & Rot, ColumnVector & pos);
void kine(Matrix & Rot, ColumnVector & pos, const int j);
ReturnMatrix kine(void);
ReturnMatrix kine(const int j);
\end{verbatim}
\subsubsection*{Description}
The forward kinematic model is provided by implementing 
the following recursion:
\begin{eqnarray}
{}^0 \mbold{R}_i & = & {}^0 \mbold{R}_{i-1} {}^{i-1}\mbold{R}_i \\
{}^0 \mbold{p}_i & = & {}^0 \mbold{p}_{i-1} + {}^0 \mbold{R}_{i-1} \mbold{p}_i 
\end{eqnarray}
where
\begin{eqnarray}
{}^0 \mbold{T}_i & = & \matr{cc}{
{}^0 \mbold{R}_{i} & {}^0 \mbold{p}_i \\
0 & 1 }
\end{eqnarray}

The overloaded function {\tt kine} can return the orientation and position or 
the equivalent homogeneous transform for the last (if not supplied) or the $i^{th}$
link. For example: 
\begin{verbatim}
Robot myrobot(init_matrix);
Matrix Thomo, R;
ColumnVector p;
/* forward kinematics up to the last link */
Thomo = myrobot.kine();  
/* forward kinematics up to the 2nd link  */
Thomo = myrobot.kine(2); 
/* forward kinematics up to the last link, outputs R and p */
myrobot.kine(R,p);       
/* forward kinematics up to the 2nd link, outputs R and p  */
myrobot.kine(R,p,2);     
\end{verbatim}
\noindent are valid calls to the function {\tt kine}.

%{\bf Note: } 

\subsubsection*{Return Value}

{\tt Matrix} or None (in this case {\tt Rot} and {\tt pos} are modified on output)

\newpage

\subsection*{kine\_pd}
\subsubsection*{Syntax}
\begin{verbatim}
ReturnMatrix kine_pd(const int ref=0);
void kine_pd(Matrix & Rot, ColumnVector & pos, 
             ColumnVector & pos_dot, const int ref=0);
\end{verbatim}
\subsubsection*{Description}   
The forward kinematic model is provided by implementing 
the following recursion (similar to \texttt{kine}):
\begin{eqnarray}
{}^0 \mbold{R}_i & = & {}^0 \mbold{R}_{i-1} {}^{i-1}\mbold{R}_i \\
{}^0 \mbold{p}_i & = & {}^0 \mbold{p}_{i-1} + {}^0 \mbold{R}_{i-1}
\mbold{p}_i 
\end{eqnarray}
\begin{eqnarray}
 \begin{array}{cc}
  {}^0 \mbold{\dot{p}}_i = {}^0 \mbold{\dot{p}}_{i-1} + {}^0
  \mbold{R}_i \mbold{\omega}_i \times {}^0 \mbold{R}_{i-1} \mbold{p}_i & \textrm{DH notation} \\
  {}^0 \mbold{\dot{p}}_i = {}^0 \mbold{\dot{p}}_{i-1} + {}^0
  \mbold{R}_{i-1} (\mbold{\omega}_{i-1} \times \mbold{p}_i) & \textrm{modified DH notation}
 \end{array}
\end{eqnarray}
where
\begin{eqnarray}
{}^0 \mbold{T}_i & = & \matr{cc}{
{}^0 \mbold{R}_{i} & {}^0 \mbold{p}_i \\
0 & 1 }
\end{eqnarray}

\subsubsection*{Return Value}
{\tt Matrix} or None (in this case {\tt Rot}, {\tt pos} {\tt pos\_dot} are modified on output)
\newpage 

\subsection*{dTdqi}
\subsubsection*{Syntax}
\begin{verbatim}
void dTdqi(Matrix & dRot, ColumnVector & dp, const int i);
ReturnMatrix dTdqi(const int i);
\end{verbatim}
\subsubsection*{Description}
This function computes the partial derivatives:
\begin{eqnarray}
\dfdx{{}^0 \mbold{T}_n}{q_i} & = & {}^0 \mbold{T}_{i-1} \mbold{Q}_i \; {}^{i-1} \mbold{T}_n
\end{eqnarray}
in standard notation and
\begin{eqnarray}
  \dfdx{{}^0 \mbold{T}_n}{q_i} 
  & = & {}^0 \mbold{T}_{i} \mbold{Q}_i \; {}^{i} \mbold{T}_n
\end{eqnarray}
in modified notation, with
\begin{eqnarray}
\mbold{Q}_i & = & \matr{cccc}{
0 & -1 & 0 & 0 \\
1 & 0 & 0 & 0 \\
0 & 0 & 0 & 0 \\
0 & 0 & 0 & 0} \mbox{ for a revolute joint} \\
\mbold{Q}_i & = & \matr{cccc}{
0 & 0 & 0 & 0 \\
0 & 0 & 0 & 0 \\
0 & 0 & 0 & 1 \\
0 & 0 & 0 & 0} \mbox{ for a prismatic joint} 
\end{eqnarray}

%{\bf Note: } 

\subsubsection*{Return Value}

{\tt Matrix} or None (in this case {\tt dRot} and {\tt dp} are modified on output)

\newpage

\subsection{Dynamics}

The robotics manipulator dynamic model is given by (see
appendix~\ref{sec:rnea} or \cite{craig})
\begin{eqnarray}
\mbold{\tau} & = & \mbold{D(q)} \ddot{\mbold{q}} + \mbold{C(q,}\dot{\mbold{q}}) 
+ \mbold{G(q)}
\end{eqnarray}

\subsection*{acceleration}
\subsubsection*{Syntax}
\begin{verbatim}
ReturnMatrix acceleration(const ColumnVector & q, 
                          const ColumnVector & qp,
                          const ColumnVector & tau);

ReturnMatrix acceleration(const ColumnVector & q, 
                          const ColumnVector & qp,
                          const ColumnVector & tau_cmd, 
                          const ColumnVector & Fext,
                          const ColumnVector & Next)
\end{verbatim}
\subsubsection*{Description}
This function computes $\ddot{\mbold{q}}$ from $\mbold{q}$, $\dot{\mbold{q}}$ and $\mbold{\tau}$ 
which is the forward dynamics problem.
Walker and Orin \cite{Walker82} presented methods to compute the inverse dynamics. 
A simplified RNE version computing 
\begin{eqnarray}
\mbold{\tau} & = & \mbold{D(q)} \ddot{\mbold{q}}
\end{eqnarray}
is implemented in the function {\tt torque\_novelocity}. By evaluating this equation $n$ times, 
one can compute $\mbold{D(q)}$ (the {\tt inertia} function), use the full RNE to compute
$\mbold{C(q,}\dot{\mbold{q}}) + \mbold{G(q)}$ and then solve the equation : 
\begin{eqnarray}
\ddot{\mbold{q}} & = & \mbold{D^{-1}(q)} \left [ \mbold{\tau} - \mbold{C(q,}\dot{\mbold{q}}) 
- \mbold{G(q)} \right ]
\end{eqnarray}

%{\bf Note: } 

\subsubsection*{Return Value}

{\tt ColumnVector}

\newpage

\subsection*{inertia}
\subsubsection*{Syntax}
\begin{verbatim}
ReturnMatrix inertia(const ColumnVector & q);
\end{verbatim}
\subsubsection*{Description}
This function computes the robot inertia matrix $\mbold{D(q)}$. A simplified RNE version computing 
\begin{eqnarray}
\mbold{\tau} & = & \mbold{D(q)} \ddot{\mbold{q}}
\end{eqnarray}
is implemented in the function {\tt torque\_novelocity}. By evaluating this equation $n$ times, 
one can compute $\mbold{D(q)}$.

%{\bf Note: } 

\subsubsection*{Return Value}

{\tt Matrix}

\newpage

\subsection*{torque}
\subsubsection*{Syntax}
\begin{verbatim}
ReturnMatrix torque(const ColumnVector & q, 
                    const ColumnVector & qp,
                    const ColumnVector & qpp);

ReturnMatrix torque(const ColumnVector & q,
                    const ColumnVector & qp,
                    const ColumnVector & qpp,
                    const ColumnVector & Fext,
                    const ColumnVector & Next);
\end{verbatim}
\subsubsection*{Description}

This function computes $\mbold{\tau}$ from $\mbold{q}$,
$\dot{\mbold{q}}$ and $\ddot{\mbold{q}}$ which is the inverse dynamics
problem. The recursive Newton-Euler (RNE) formulation is one of the
most computationally efficient algorithm \cite{Luh80,Murray86} used to
solve this problem (see appendix~\ref{sec:rnea}). The second form
allows the inclusion the contribution of a load applied at the last link.

%{\bf Note: } 

\subsubsection*{Return Value}

{\tt ColumnVector}

\newpage

\subsection*{torque\_novelocity}
\subsubsection*{Syntax}
\begin{verbatim}
ReturnMatrix torque_novelocity(const ColumnVector & q,
                               const ColumnVector & qpp);

ReturnMatrix torque_novelocity(const ColumnVector & q,
                               const ColumnVector & qpp,
                               const ColumnVector & Fext,
                               const ColumnVector & Next);
\end{verbatim}
\subsubsection*{Description}
This function computes $\mbold{\tau}$ from $\mbold{q}$ 
and $\ddot{\mbold{q}}$ when $\dot{\mbold{q}} = 0$ and gravity is set to zero.

%{\bf Note: } 

\subsubsection*{Return Value}

{\tt ColumnVector}

\newpage

\subsection*{G and C}
\subsubsection*{Syntax}
\begin{verbatim}
ReturnMatrix G();
ReturnMatrix C();
\end{verbatim}
\subsubsection*{Description}
The function G() computes $\mbold{\tau}$ from the gravity effect,
while C() computes $\mbold{\tau}$ from the Coriolis and centrifugal
effects.

\subsubsection*{Return Value}

{\tt ColumnVector} for G and C

\newpage

\subsection{Linearized dynamics}
Murray and Neuman \cite{Murray86} have developed an efficient recursive linearized Newton-Euler formulation 
that can be used to compute (see appendix~\ref{sec:rnea})
\begin{eqnarray}
\delta \mbold{\tau} & = & \mbold{D(q)} \delta \ddot{\mbold{q}} 
+ \mbold{S_1(q,}\dot{\mbold{q}}) \delta \dot{\mbold{q}}
+ \mbold{S_2(q,}\dot{\mbold{q}},\ddot{\mbold{q}}) \delta \mbold{q}
\end{eqnarray}

\subsection*{delta\_torque}
\subsubsection*{Syntax}
\begin{verbatim}
void delta_torque(const ColumnVector & q, 
                  const ColumnVector & qp,
                  const ColumnVector & qpp, 
                  const ColumnVector & dq,
                  const ColumnVector & dqp, 
                  const ColumnVector & dqpp,
                  ColumnVector & torque, 
                  ColumnVector & dtorque);
\end{verbatim}
\subsubsection*{Description}
This function computes 
\begin{eqnarray}
\delta \mbold{\tau} & = & \mbold{D(q)} \delta \ddot{\mbold{q}} 
+ \mbold{S_1(q,}\dot{\mbold{q}}) \delta \dot{\mbold{q}}
+ \mbold{S_2(q,}\dot{\mbold{q}},\ddot{\mbold{q}}) \delta \mbold{q}
\end{eqnarray}

%{\bf Note: } 

\subsubsection*{Return Value}

None ({\tt torque} and {\tt dtorque} are modified on output)

\newpage

\subsection*{dq\_torque}
\subsubsection*{Syntax}
\begin{verbatim}
void dq_torque(const ColumnVector & q, 
               const ColumnVector & qp,
               const ColumnVector & qpp, 
               const ColumnVector & dq,
               ColumnVector & torque, 
               ColumnVector & dtorque);
\end{verbatim}
\subsubsection*{Description}
This function computes 
\begin{eqnarray}
\mbold{S_2(q,}\dot{\mbold{q}},\ddot{\mbold{q}}) \delta \mbold{q}
\end{eqnarray}

%{\bf Note: } 

\subsubsection*{Return Value}

None ({\tt torque} and {\tt dtorque} are modified on output)

\newpage
\subsection*{dqp\_torque}
\subsubsection*{Syntax}
\begin{verbatim}
void dqp_torque(const ColumnVector & q, 
                const ColumnVector & qp,
                const ColumnVector & dqp, 
                ColumnVector & torque,
                ColumnVector & dtorque);
\end{verbatim}
\subsubsection*{Description}
This function computes 
\begin{eqnarray}
\mbold{S_1(q,}\dot{\mbold{q}}) \delta \dot{\mbold{q}}
\end{eqnarray}

%{\bf Note: } 

\subsubsection*{Return Value}

None ({\tt torque} and {\tt dtorque} are modified on output)

\newpage

\subsection*{dtau\_dq}
\subsubsection*{Syntax}
\begin{verbatim}
ReturnMatrix dtau_dq(const ColumnVector & q, 
                     const ColumnVector & qp,
                     const ColumnVector & qpp);
\end{verbatim}
\subsubsection*{Description}
This function computes 
\begin{eqnarray}
\dfdx{\mbold{\tau}}{\mbold{q}} & = & \mbold{S_2(q,}\dot{\mbold{q}},\ddot{\mbold{q}})
\end{eqnarray}

%{\bf Note: } 

\subsubsection*{Return Value}

{\tt Matrix}

\newpage

\subsection*{dtau\_dqp}
\subsubsection*{Syntax}
\begin{verbatim}
ReturnMatrix dtau_dqp(const ColumnVector & q, 
                      const ColumnVector & qp);
\end{verbatim}
\subsubsection*{Description}
This function computes 
\begin{eqnarray}
\dfdx{\mbold{\tau}}{\dot{\mbold{q}}} & = & \mbold{S_1(q,}\dot{\mbold{q}}) 
\end{eqnarray}

%{\bf Note: } 

\subsubsection*{Return Value}

{\tt Matrix}

\newpage

\subsection*{perturb\_robot}
\subsubsection*{Syntax}
\begin{verbatim}
void perturb_robot(Robot_basic & robot, const double f = 0.1);
\end{verbatim}
\subsubsection*{Description}
This function, which is not a member of any class, modifies randomly
the robot parameters. The parameter variation in percentage is
described by \texttt{f}.

\subsubsection*{Return Value}

None

\newpage

\section{The \texttt{Spl\_Cubic} class}

Spl\_Cubic deals with parametric cubic splines \cite{Angeles97}.

\subsection*{Constructor}
\subsubsection*{Syntax}
\begin{verbatim}
Spl_cubic(){};
Spl_cubic(const Matrix & pts);
Spl_cubic(const Spl_cubic & x);
Spl_cubic & operator=(const Spl_cubic & x);
\end{verbatim}

\subsubsection*{Description}   
Spl\_Cubic object constructor, copy constructor and equal operator.

\subsubsection*{Return Value}

None

\newpage

\subsection*{s, ds and dds}
\subsubsection*{Syntax}
\begin{verbatim}
short interpolating(const Real t, ColumnVector & s);
short first_derivative(const Real t, ColumnVector & ds);
short second_derivative(const Real t, ColumnVector & dds);
\end{verbatim}

\subsubsection*{Description}   
These functions interpolate the spline at time $t$ to sets the
quaternion $s$, $ds$ and $dds$.

\subsubsection*{Return Value}

Status, as a short int.
\begin{enumerate}
\item[] 0 successful
\item[] NOT\_IN\_RANGE (regarding $t$)
\item[] BAD\_DATA
\end{enumerate}


\newpage

\section{The \texttt{Spl\_path} class}

Spl\_path uses three instances of the class Spl\_Cubic for path $X$,
$Y$, $Z$ interpolation.

\subsection*{Constructor}
\subsubsection*{Syntax}
\begin{verbatim}
Spl_path():Spl_cubic(){};
Spl_path(const string & filename);
Spl_path(const Matrix & x);
Spl_path(const Spl_path & x);
Spl_path & operator=(const Spl_path & x);
\end{verbatim}

\subsubsection*{Description}   
Spl\_path object constructor, copy constructor and equal operator.

\subsubsection*{Return Value}

None

\newpage

\subsection*{p, dp, ddp}
\subsubsection*{Syntax}
\begin{verbatim}
short p(const Real time, ColumnVector & p);
short p_pdot(const Real time, ColumnVector & p, ColumnVector & pdot);
short p_pdot_pddot(const Real time, ColumnVector & p, ColumnVector & dp, 
                   ColumnVector & ddp);
\end{verbatim}

\subsubsection*{Description}   
These functions interpolate the spline at time $t$ to sets the
quaternion $p$ (position), $dp$ (velocity) and $ddp$ (acceleration).

\subsubsection*{Return Value}

Status, as a short int.
\begin{enumerate}
\item[] 0 successful
\item[] NOT\_IN\_RANGE (regarding $t$)
\item[] BAD\_DATA
\end{enumerate}

\newpage

\section{The \texttt{Spl\_Quaternion} class}

Spl\_Quaternion deals with parametric quaternions cubic splines.

\subsection*{Constructor}
\subsubsection*{Syntax}
\begin{verbatim}
Spl_Quaternion(){}
Spl_Quaternion(const string & filename);
Spl_Quaternion(const quat_map & quat);
Spl_Quaternion(const Spl_Quaternion & x);
Spl_Quaternion & operator=(const Spl_Quaternion & x);
\end{verbatim}

\subsubsection*{Description}   
Spl\_Quaternion object constructor, copy constructor and equal operator.

\subsubsection*{Return Value}

None

\newpage

\subsection*{quat and quat\_w}
\subsubsection*{Syntax}
\begin{verbatim}
short quat(const Real t, Quaternion & q);
short quat_w(const Real t, Quaternion & q, ColumnVector & w);
\end{verbatim}

\subsubsection*{Description}   
These functions interpolate the spline at time $t$ to sets the
quaternion $q$ and the angular velocity $\omega$.

\subsubsection*{Return Value}

Status, as a short int.
\begin{enumerate}
\item[] 0 successful
\item[] NOT\_IN\_RANGE (regarding $t$)
\end{enumerate}

\newpage

\section{The \texttt{Trajectory\_Select} class}

This class deals with trajectory selection logic.

\subsection*{Constructor}
\subsubsection*{Syntax}
\begin{verbatim}
Trajectory_Select();
Trajectory_Select(const string & filename);
Trajectory_Select(const Trajectory_Select & x);
Trajectory_Select & operator=(const Trajectory_Select & x);
\end{verbatim}

\subsubsection*{Description}   
Trajectory\_Select object constructor, copy constructor and equal operator.

\subsubsection*{Return Value}

None

\newpage

\subsection*{set\_trajectory}
\subsubsection*{Syntax}
\begin{verbatim}
void set_trajectory(const string & filename);
\end{verbatim}

\subsubsection*{Description}   
This function reads the trajectory file (filename) and 
assign the spline data in class Spl\_path or in class
Spl\_Quaternion.

\subsubsection*{Return Value}

None

\newpage

\section{The \texttt{CLIK} class}

The \emph{CLICK} class deals with closed-loop inverse kinematics
algorithm based on the unit quaternion \cite{Chiaverini99}.

\subsection*{Constructor}
\subsubsection*{Syntax}
\begin{verbatim}
Clik(){}
Clik(const Robot & robot_, const Matrix & Kp_, const Matrix & Ko_, 
     const Real eps_=0.04, const Real lambda_max_=0.04, 
     const Real dt=1.0);
Clik(const mRobot & mrobot_, const Matrix & Kp_, const Matrix & Ko_, 
     const Real eps_=0.04, const Real lambda_max_=0.04, 
     const Real dt=1.0);
Clik(const mRobot_min_para & mrobot_min_para_, const Matrix & Kp_, 
     const Matrix & Ko_, const Real eps_=0.04, 
     const Real lambda_max_=0.04, const Real dt=1.0);
Clik(const Clik & x);
Clik & operator=(const Clik & x);
\end{verbatim}

\subsubsection*{Description}   
\emph{CLIK} object constructor, copy constructor and equal operator.

\subsubsection*{Return Value}

None

\newpage

\subsection*{q\_qdot}
\subsubsection*{Syntax}
\begin{verbatim}
void q_qdot(const Quaternion & qd, const ColumnVector & pd, 
            const ColumnVector & pddot, const ColumnVector & wd,
            ColumnVector & q, ColumnVector & qp);
\end{verbatim}

\subsubsection*{Description}   
This function sets the desired orientation joint position $q$ and the
desired joint velocity $qp$.
\subsubsection*{Return Value}

None

\newpage

\section{The \texttt{Proportional\_Derivative} class}

The \emph{Proportional\_Derivative} class deals with the well known
proportional derivative position controller.

\subsection*{Constructor}
\subsubsection*{Syntax}
\begin{verbatim}
Proportional_Derivative(const short dof = 1);
Proportional_Derivative(const Robot_basic & robot, const DiagonalMatrix & Kp, 
                        const DiagonalMatrix & Kd);
Proportional_Derivative(const Proportional_Derivative & x);
\end{verbatim}

\subsubsection*{Description}   
\emph{Proportional\_Derivative} object constructor, copy constructor
and equal operator.

\subsubsection*{Return Value}

None

\newpage

\subsection*{torque\_cmd}
\subsubsection*{Syntax}
\begin{verbatim}
ReturnMatrix torque_cmd(Robot_basic & robot, const ColumnVector & qd,
                        const ColumnVector & qpd);
\end{verbatim}

\subsubsection*{Description}   
This function sets the output torque for a desired joint position
vector, $q_d$, and a desired joint velocity vector, $\dot{q}_d$.

\subsubsection*{Return Value}

\texttt{Matrix}

\newpage

\subsection*{$K_d$, $K_p$}
\subsubsection*{Syntax}
\begin{verbatim}
short set_Kd(const DiagonalMatrix & Kd);
short set_Kp(const DiagonalMatrix & Kp);
\end{verbatim}

\subsubsection*{Description}   
These functions sets the joint position error gain matrix, $K_d$, and
the joint velocity error gain matrix, $K_p$.
\subsubsection*{Return Value}

Status, as a short int.
\begin{enumerate}
\item[] 0 successful
\item[] WRONG\_SIZE (regarding the input vector)
\end{enumerate}

\newpage

\section{The \texttt{Computed\_torque\_method} class}

The \emph{Computed\_torque\_method} class deals with the well known
computed torque method position controller \cite{Fu87}.

\subsection*{Constructor}
\subsubsection*{Syntax}
\begin{verbatim}
Computed_torque_method();
Computed_torque_method(const Robot_basic & robot, 
                       const DiagonalMatrix & Kd, const DiagonalMatrix & Kp);
Computed_torque_method(const Computed_torque_method & x);
Computed_torque_method & operator=(const Computed_torque_method & x);
\end{verbatim}

\subsubsection*{Description}   
\emph{Computed\_torque\_method} object constructor, copy constructor
and equal operator.

\subsubsection*{Return Value}

None

\newpage

\subsection*{torque\_cmd}
\subsubsection*{Syntax}
\begin{verbatim}
ReturnMatrix torque_cmd(Robot_basic & robot, const ColumnVector & qd,
                        const ColumnVector & qpd);
\end{verbatim}

\subsubsection*{Description}   
This function sets the output torque for a desired joint position
vector, $q_d$, and a desired joint velocity vector, $\dot{q}_d$.

\subsubsection*{Return Value}

\texttt{Matrix}

\newpage

\subsection*{$K_d$, $K_p$}
\subsubsection*{Syntax}
\begin{verbatim}
short set_Kp(const DiagonalMatrix & Kp);
short set_Kd(const DiagonalMatrix & Kd);

\end{verbatim}

\subsubsection*{Description}   
These functions sets the joint position error gain matrix, $K_p$, and
the joint velocity error gain matrix, $K_d$.
\subsubsection*{Return Value}

Status, as a short int.
\begin{enumerate}
\item[] 0 successful
\item[] WRONG\_SIZE (regarding the input vector)
\end{enumerate}

\newpage

\section{The \texttt{Resolve\_acc} class}

The \emph{Resolve\_acc} class deals with the resolve rate acceleration
controller \cite{Caccavale98}.

\subsection*{Constructor}
\subsubsection*{Syntax}
\begin{verbatim}
Resolved_acc();
Resolved_acc(const Robot_basic & robot, 
             const double Kvp, const double Kpp, 
             const double Kvo, const double Kpo);
Resolved_acc(const Resolved_acc & x);
Resolved_acc & operator=(const Resolved_acc & x);
\end{verbatim}

\subsubsection*{Description}   
\emph{Resolve\_acc} object constructor, copy constructor and equal
operator.

\subsubsection*{Return Value}

None

\newpage

\subsection*{torque\_cmd}
\subsubsection*{Syntax}
\begin{verbatim}
ReturnMatrix torque_cmd(Robot_basic & robot, const ColumnVector & pdpp, 
                        const ColumnVector & pdp, const ColumnVector & pd, 
                        const ColumnVector & wdp, const ColumnVector & wd, 
                        const Quaternion & qd, const short link_pc, 
                        const Real dt);
\end{verbatim}

\subsubsection*{Description}   
This function sets the output torque for the following desired end
effector vector: acceleration, velocity, position, angular
acceleration, angular velocity and angular position.
\subsubsection*{Return Value}

\texttt{Matrix}

\newpage

\subsection*{$K_{pp}$, $K_{vp}$, $K_{po}$, $K_{vo}$}
\subsubsection*{Syntax}
\begin{verbatim}
void set_Kpp(const double Kpp);
void set_Kvp(const double Kvp);
void set_Kpo(const double Kpo);
void set_Kvo(const double Kvo);
\end{verbatim}

\subsubsection*{Description}   
These functions sets the end effector position error gain, $K_{pp}$, the velocity
error gain, $K_{vp}$, the orientation error gain $K_{po}$, and the
orientation angular rate gain, $K_{vo}$.
\subsubsection*{Return Value}

None


\newpage

\section{The \texttt{Impedance} class}

The \emph{Impedance} class deals with the impedance controller
\cite{Caccavale99}. This class should be use with the class
\emph{Resolve\_acc}.  \emph{Resolve\_acc} will make sure the end
effector follow the compliant trajectory generated by
\emph{Impedance}. The end effector impedance is defined in terms of
its translational and rotational part \cite{Caccavale99}.

\subsection*{Constructor}
\subsubsection*{Syntax}
\begin{verbatim}
Impedance();
Impedance(const Robot_basic & robot, const DiagonalMatrix & Mp, 
          const DiagonalMatrix & Dp, const DiagonalMatrix & Kp, 
          const Matrix & Km,         const DiagonalMatrix & Mo, 
          const DiagonalMatrix & Do, const DiagonalMatrix & Ko);
Impedance(const Impedance & x);
Impedance & operator=(const Impedance & x);
\end{verbatim}

\subsubsection*{Description}   
\emph{Impedance} object constructor, copy constructor and equal
operator.

\subsubsection*{Return Value}

None

\newpage

\subsection*{control}
\subsubsection*{Syntax}
\begin{verbatim}
short control(const ColumnVector & pdpp, const ColumnVector & pdp,
              const ColumnVector & pd, const ColumnVector & wdp,
              const ColumnVector & wd, const Quaternion & qd, 
              const ColumnVector & f, const ColumnVector & n,
              const Real dt); 
\end{verbatim}

\subsubsection*{Description}   
This function generate the compliant trajectory for a desired
trajectory.


\subsubsection*{Return Value}

Status, as a short int.
\begin{enumerate}
\item[] 0 successful
\item[] WRONG\_SIZE (regarding the input vector)
\end{enumerate}

\newpage

\subsection*{$M_p$, $D_p$, $K_p$, $M_o$, $D_o$, $K_o$}
\subsubsection*{Syntax}
\begin{verbatim}
short set_Mp(const DiagonalMatrix & Mp);
short set_Mp(Real MP_i, const short i);
short set_Dp(const DiagonalMatrix & Dp);
short set_Dp(Real Dp_i, const short i);
short set_Kp(const DiagonalMatrix & Kp);
short set_Kp(Real Kp_i, const short i);
short set_Mo(const DiagonalMatrix & Mo);
short set_Mo(Real Mo_i, const short i);
short set_Do(const DiagonalMatrix & Do);
short set_Do(Real Do_i, const short i);
short set_Ko(const DiagonalMatrix & Ko);
short set_Ko(Real Ko_i, const short i);
\end{verbatim}

\subsubsection*{Description}   
These functions sets the translational and rotational impedance
parameters.
\subsubsection*{Return Value}

Status, as a short int.
\begin{enumerate}
\item[] 0 successful
\item[] WRONG\_SIZE (regarding the input vector)
\end{enumerate}


\newpage

\section{The \texttt{Control\_Select} class}

The \emph{Control\_Select} class deals with the controllers
selection logic. It can be use to select any controllers mentioned
above by reading the input file.
\subsection*{Constructor}
\subsubsection*{Syntax}
\begin{verbatim}
Control_Select();
Control_Select(const string & filename);
Control_Select(const Control_Select & x);
Control_Select & operator=(const Control_Select & x);
\end{verbatim}

\subsubsection*{Description}   
\emph{Control\_Select} object constructor, copy constructor and equal
operator.

\subsubsection*{Return Value}

None

\newpage

\subsection*{get\_dof}
\subsubsection*{Syntax}
\begin{verbatim}
int get_dof();
\end{verbatim}

\subsubsection*{Description}   
This function return the degree of freedom used in the selection.
\subsubsection*{Return Value}

\texttt{int}

\newpage

\subsection*{set\_control}
\subsubsection*{Syntax}
\begin{verbatim}
void set_control(const string & filename);
\end{verbatim}

\subsubsection*{Description}   
This function set the active controller.
\subsubsection*{Return Value}

None

\newpage

\section{The \texttt{Stewart} class}

Coming soon ... (based on \cite{Harib03}).

\newpage

\section{The \texttt{IO\_matrix\_file} class}

Read and write functions are provided by the class IO\_matrix\_file.
It is possible to read or write data at every iteration of the
simulation using an instance of this class.

\subsection*{Constructor}
\subsubsection*{Syntax}
\begin{verbatim}
IO_matrix_file(const string & filename);
\end{verbatim}

\subsubsection*{Description}   
IO\_matrix\_file object constructor.

\subsubsection*{Return Value}

None

\newpage

\subsection*{write}
\subsubsection*{Syntax}
\begin{verbatim}
short write(const vector<Matrix> & data);
short write(const vector<Matrix> & data, const vector<string> & data_title);
\end{verbatim}

\subsubsection*{Description}   
This member function appends \texttt{data} to a file (specified by the
constructor, and opened by write() when first called). 
\texttt{data\_title} is used to write a header description at the
beginning of the file. If it is not specified, a default 
description $datai, i=1,2,\cdots,n$ will be added. The header
contains the number of iterations,
the number of vectors and the data parameters, as follows:

\indent
\emph{nb$\_$iterations 1269}
 
\emph{nb$\_$vector 2} 

\emph{nb$\_$rows 1  nb$\_$cols 1  time (s)}

\emph{nb$\_$rows 6  nb$\_$cols 1  q(i) (rad)}

---------------------------------

%\noindent
\subsubsection*{Return Value}
A short integer return the status:
\begin{enumerate}
\item[] 0 successful, 
\item[] IO\_COULD\_NOT\_OPEN\_FILE
\item[] IO\_DATA\_EMPTY
\end{enumerate}
\newpage 

\subsection*{read}
\subsubsection*{Syntax}
\begin{verbatim}
short read(const vector<Matrix> & data);
short read(const vector<Matrix> & data, const vector<string> & data_title);
short read_all(vector<Matrix> & data, vector<string> & data_title);
\end{verbatim}

\subsubsection*{Description}   
These member functions read \texttt{data} from a file (specified by
the constructor, and opened when first called). \texttt{read()} reads
the values corresponding to only one iteration, while
\texttt{read\_all()} reads the entire file at once.

\noindent
These member functions are meant to read a file that was written using
\texttt{write()}.

\subsubsection*{Return Value}
Status, as a short int.
\begin{enumerate}
\item[] 0 successful
\item[] IO\_DATA\_EMPTY
\item[] IO\_COULD\_NOT\_OPEN\_FILE
\end{enumerate}

\newpage 

\section{Graphics}

Graphics are provided through calls to the \textsf{gnuplot} \footnote{
  \textsf{gnuplot} is freely available from the following location:
  {\tt
    \htmladdnormallink{http://www.gnuplot.info/}{http://www.gnuplot.info/}}
  } software. Instances of the class {\tt Plot2d} and {\tt Plot\_file} are used 
to generate the data and command files required by the call to \textsf{gnuplot}. 
A plot can be generated using the \texttt{set\_plot2d} function.
\newpage 

\subsection*{\texttt{Plot2d} class}
\subsection*{Constructor}
 \subsubsection*{Syntax}
 \begin{verbatim}
  Plot2d(void);
 \end{verbatim}

 \subsubsection*{Description}   
 Upon initialization, a {\tt Plot2d} object contain an empty graph.
 Data, title, label and other goodies can be added using the following
 member functions:
\begin{itemize}
\item {\tt addcommand};
\item {\tt addcurve};
\item {\tt dump};
\item {\tt gnuplot};
\item {\tt settitle};
\item {\tt setxlabel};
\item {\tt setylabel}.
\end{itemize}

 %{\bf Note: }

\subsubsection*{Return Value}

{\tt None}
\newpage

\subsection*{addcommand}
\subsubsection*{Syntax}
\begin{verbatim}
void addcommand(const char * gcom);
\end{verbatim}
\subsubsection*{Description}
This function adds the command specified by the string {\tt gcom} to 
the \textsf{gnuplot} command file. Ex: {\tt mygraph.addcommand("set grid")}.

\noindent {\bf Note: } see the \textsf{gnuplot} documentation for the list
of commands.

\subsubsection*{Return Value}

None 

\newpage

\subsection*{addcurve}
\subsubsection*{Syntax}
\begin{verbatim}
void addcurve(const Matrix & data, 
              const char * label = "", 
              int type = LINES);
\end{verbatim}
\subsubsection*{Description}
This function add the curves specified by the $n \times 2$ matrix
{\tt data} to the plot using the string {\tt label} for the legend
and {\tt type} for the curve line type. Defined line types are:
\begin{itemize}
\item LINES;
\item POINTS;
\item LINESPOINTS;
\item IMPULSES;
\item DOTS;
\item STEPS;
\item BOXES.
\end{itemize}
See the \textsf{gnuplot} documentation for the description of these line
types.

%\noindent {\bf Note: } 

\subsubsection*{Return Value}

None

\newpage

\subsection*{dump}
\subsubsection*{Syntax}
\begin{verbatim}
void dump(void);
\end{verbatim}
\subsubsection*{Description}
This function dumps the current content of the object to {\tt stdout}.
%{\bf Note: } 

\subsubsection*{Return Value}

None

\newpage

\subsection*{gnuplot}
\subsubsection*{Syntax}
\begin{verbatim}
void gnuplot(void);
\end{verbatim}
\subsubsection*{Description}
This function calls \textsf{gnuplot} with the current content of the object.
%{\bf Note: } 

\subsubsection*{Return Value}

None

\newpage

\subsection*{settitle}
\subsubsection*{Syntax}
\begin{verbatim}
void settitle(const char * t);
\end{verbatim}
\subsubsection*{Description}
This function sets the title of the graph to the string {\tt t}.
%{\bf Note: } 

\subsubsection*{Return Value}

None

\newpage

\subsection*{setxlabel}
\subsubsection*{Syntax}
\begin{verbatim}
void setxlabel(const char * t);
\end{verbatim}
\subsubsection*{Description}
This function sets the axis X label of the graph to the string {\tt t}.
%{\bf Note: } 

\subsubsection*{Return Value}

None

\newpage

\subsection*{setylabel}
\subsubsection*{Syntax}
\begin{verbatim}
void setylabel(const char * t);
\end{verbatim}
\subsubsection*{Description}
This function sets the axis Y label of the graph to the string {\tt t}.
%{\bf Note: } 

\subsubsection*{Return Value}

None

\newpage

\subsection*{\texttt{Plot\_file} class}
An instance of this class allows the creation of graphics from a data
file. This file has to be created with an instance of the class IO\_matrix\_file. 
\subsection*{Constructor}
\subsubsection*{Syntax}
\begin{verbatim}
Plot_file(const string & filename);
\end{verbatim}

\subsubsection*{Description}   
Plot\_file object constructor.

\subsubsection*{Return Value}

{\tt None}
 \newpage 

\subsection*{graph}
\subsubsection*{Syntax}
\begin{verbatim}
short graph(const string & title_graph, const string & label, const short x,
            const short y, const short x_start, const short y_start,
            const short y_end);
\end{verbatim}

\subsubsection*{Description}   

Create a graphic from a data file (specified by constructor).
\texttt{title\_graph} and \texttt{label} are used to provide the
graphic title and label names in the legend. \texttt{x} refers to the
index in the ``\texttt{vector$<$Matrix$>$ \& data}'' (in class
\texttt{IO\_Matrix\_file}) corresponding to the \texttt{x} axis (ex:
time), while \texttt{y} refers to the index in the
``\texttt{vector$<$Matrix$>$ \& data}'' corresponding to the
\texttt{y} axis (ex: joints positions). \texttt{x\_start},
\texttt{y\_start} and \texttt{y\_end} specify which rows of
\texttt{data} to use.


\subsubsection*{Return Value}
Status, as a short int.
\begin{enumerate}
\item[] 0 successful
\item[] X\_Y\_DATA\_NO\_MATCH
\item[] PROBLEM\_FILE\_READING
\end{enumerate}

 \newpage 

\subsection*{set\_plot2d}
\subsubsection*{Syntax}
\begin{verbatim}
void set_plot2d(const char *title_graph, const char *x_axis_title, 
                const char *y_axis_title, const char *label, int type, 
                const Matrix &xdata, const Matrix &ydata,
                int start_y, int end_y);

void set_plot2d(const char *title_graph, const char *x_axis_title, 
                const char *y_axis_title, const std::vector<char *> label,
                int type, const Matrix &xdata, const Matrix &ydata,
                const std::vector<int> & data_select);
\end{verbatim}
\subsubsection*{Description}
This function generates a plot using a range (\texttt{start\_y,
  end\_y}) or a selection of columns (\texttt{data\_select}) of the
\texttt{ydtata} while setting the titles and labels.
%{\bf Note: } 

\subsubsection*{Return Value}

None

\newpage

\section{Config class}
\subsection*{Config}
 \subsubsection*{Syntax}
 \begin{verbatim}
  Config(const string & filename, const bool bPrintErrorMessages = true);
  Config(const Config & x);
  Config & operator=(const Config & x);
 \end{verbatim}

\subsubsection*{Description}   
This class provides a function to read a configuration. 

\subsubsection*{Return Value}

{\tt None}
 \newpage 

\subsection*{Reading and writing}
\subsubsection*{Syntax}
\begin{verbatim}
short read_conf();
short write_conf(const string filename, const string file_title,
                 const int space_between_column);
\end{verbatim}

\subsubsection*{Description}   
The member function read\_conf reads a configuration file (specified
by constructor). The member function \emph{write\_conf} writes the
configuration data in a file. A configuration file is divided in
sections, which contain different parameters with their values. A
section starts by $[$\emph{section\_name}$]$ and contains one or more
parameters an their values: \emph{parameter\_name: value} The ``:'' is
mandatory between the name of the parameter and it's value. Lines
beginning with a \# and white/empty lines are ignored .  The following
example contains one section named \emph{PUMA560\_mDH}.

\begin{verbatim}
[PUMA560_mDH]
DH:         0
Fix:        1
MinPara:    0
dof:        6
Motor:      0
\end{verbatim}

\subsubsection*{Return Value}
Status, as a short int.
\begin{enumerate}
\item[] 0 successful
\item[] CAN\_NOT\_OPEN\_FILE
\end{enumerate}
\newpage 

\subsection*{select}
\subsubsection*{Syntax}
\begin{verbatim}
short select(const string section, const string parameter,
                    string & value) const;
short select(const string section, const string parameter,
                  bool & value) const;
short select(const string section, const string parameter,
                   short & value) const;
short select(const string section, const string parameter,
                 int & value) const;
short select(const string section, const string parameter,
                    double & value) const;
short select(const string section, const string parameter,
                    float & value) const;
\end{verbatim}

\subsubsection*{Description}   
These member functions are use to assign to the variable
\texttt{value} the value of the parameter \texttt{parameter} from
section section \texttt{section}.

\subsubsection*{Return Value}
Status, as a short int.
\begin{enumerate}
\item[] 0 successful
\item[] SECTION\_OR\_PARAMETER\_DOES\_NOT\_EXIST
\end{enumerate}
 \newpage 

\subsection*{add}
\subsubsection*{Syntax}
\begin{verbatim}
void add(const string section, const string parameter,
                const string value);
void add(const string section, const string parameter,
              const bool value);
void add(const string section, const string parameter,
               const short value);
void add(const string section, const string parameter,
             const int value);
void add(const string section, const string parameter,
                const double value);
void add(const string section, const string parameter,
                const float value);
\end{verbatim}

\subsubsection*{Description}   
These member functions are use to add data into the data file
structure. They will create the section and the parameter if it does
not already exist.

\subsubsection*{Return Value}
None
\newpage 

\section{Miscellaneous}

\subsection*{odeint}
\subsubsection*{Syntax}
\begin{verbatim}
void odeint(ReturnMatrix (*xdot)(Real time, const Matrix & xin),
            Matrix & xo, 
            Real to, 
            Real tf, 
            Real eps, 
            Real h1, 
            Real hmin,
            int & nok, 
            int & nbad,
            RowVector & tout, 
            Matrix & xout, 
            Real dtsav);
\end{verbatim}
\subsubsection*{Description}
This function performs the numerical integration of 
\begin{eqnarray}
\dot{\mbold{x}} & = & \mbold{f}(\mbold{x}(t),t)
\end{eqnarray}
using an adaptive step size based on $4^{th}$ order Runge-Kutta scheme.
It carries out the integration of {\tt xdot} with the initial conditions given by
{\tt xo}, from time {\tt to} to {\tt tf} with accuracy {\tt eps} saving the results at
{\tt dtsav} increments. After the function call, 
{\tt tout} is set as 
\begin{eqnarray}
\matr{cccc}{ t_0 & t_1 & \cdots & t_{nsteps}}
\end{eqnarray}
{\tt xout} as
\begin{eqnarray}
\matr{cccc}{ \mbold{x}_0 & \mbold{x}_1 & \cdots & \mbold{x}_{nsteps}}
\end{eqnarray}
{\tt xo} as $\mbold{x}_{nsteps}$, {\tt nok} and {\tt nbad} to the number of
good and bad steps taken. The function {\tt odeint} is adapted from \cite{press_flannery}.


%\noindent {\bf Note: } 

\subsubsection*{Return Value}

None ({\tt xo}, {\tt tout} and {\tt xout} are modified on output)

\newpage

\subsection*{Runge\_Kutta4}
\subsubsection*{Syntax}
\begin{verbatim}
void Runge_Kutta4(ReturnMatrix (*xdot)(Real time, const Matrix & xin),
                  const Matrix & xo, 
                  Real to, 
                  Real tf, 
                  int nsteps,
                  RowVector & tout, 
                  Matrix & xout);
\end{verbatim}
\subsubsection*{Description}
This function performs the numerical integration of 
\begin{eqnarray}
\dot{\mbold{x}} & = & \mbold{f}(\mbold{x}(t),t)
\end{eqnarray}
using a fixed step size $4^{th}$ order Runge-Kutta scheme.  It carries
out the integration of {\tt xdot} with the initial conditions given by
{\tt xo}, from time {\tt to} to {\tt tf} with {\tt nsteps}. After the
function call, {\tt tout} is set as
\begin{eqnarray}
\matr{cccc}{ t_0 & t_1 & \cdots & t_{nsteps}}
\end{eqnarray}
and {\tt xout} as
\begin{eqnarray}
\matr{cccc}{ \mbold{x}_0 & \mbold{x}_1 & \cdots & \mbold{x}_{nsteps}}
\end{eqnarray}


%\noindent {\bf Note: } 

\subsubsection*{Return Value}

None ({\tt tout} and {\tt xout} are modified on output)

\newpage

\subsection*{Integ\_Trap}
\subsubsection*{Syntax}
\begin{verbatim}
ReturnMatrix Integ_Trap(const ColumnVector & present, ColumnVector & past, 
                        Real dt);
\end{verbatim}
\subsubsection*{Description}
This function performs the trapezoidal integration of the vector
$\mbold{past}$ to vector $\mbold{present}$ over $\mbold{dt}$.


\subsubsection*{Return Value}

{\tt Matrix}

\newpage

\subsection*{pinv}
\subsubsection*{Syntax}
\begin{verbatim}
ReturnMatrix pinv(const Matrix & M);
\end{verbatim}
\subsubsection*{Description}
This function computes the pseudo inverse of the matrix $M$ using SVD.
If $ A = U^{*}QV $ is a singular value decomposition of $A$, then $
A^{\dagger} = V^{*}Q^{\dagger}U$ where $ X^{*} $ is the conjugate
transpose of $ X $ and
\begin{eqnarray*}
  Q^{\dagger} = \left [
    \begin{array}{cccc}
      1/\sigma_1 & & & \\
      & 1/\sigma_2& & \\
      &  & \ddots & \\
      &  &  & 0
    \end{array}
  \right ]
\end{eqnarray*}
where the $1/\sigma_i $ are replaced by 0 when $1/\sigma_i < tol $.


\subsubsection*{Return Value}

{\tt Matrix}
\newpage

\subsection*{vec\_dot\_prod}
\subsubsection*{Syntax}
\begin{verbatim}
Real vec_dot_prod(const ColumnVector & x, const ColumnVector & y);
\end{verbatim}
\subsubsection*{Description}
This function performs the vector dot product on {\tt x} and {\tt y}.
%{\bf Note: } 

\subsubsection*{Return Value}

{\tt ColumnVector}

\newpage

\subsection*{x\_prod\_matrix}
\subsubsection*{Syntax}
\begin{verbatim}
ReturnMatrix x_prod_matrix(const ColumnVector & x);
\end{verbatim}
\subsubsection*{Description}
This function computes the cross product matrix $S(x)$ of {\tt x} such
that $S(x)y = x \times y$.
%{\bf Note: } 

\subsubsection*{Return Value}

{\tt Matrix}
\newpage

\section{Summary of functions}

\begin{table}[htbp]
\caption{Homogeneous transforms}
\label{tab:commandsum1}
\begin{center}
\begin{tabular}{||l|l||}
\hline
\hline
\multicolumn{2}{||l||}{{\bf Homogeneous Transforms}} \\
\hline
\hline
eulzxz & transform of Euler angles \\
\hline
ieulzxz & Euler angles of a transform \\
\hline
irotk & rotation around a unit vector of a transform \\
\hline
irpy & roll-pitch-yaw angles of a transform \\
\hline
rotd & transform of a rotation around a line segment \\
\hline
rotk & transform of a rotation around a unit vector \\
\hline
rpy & transform of roll-pitch-yaw angles \\
\hline
rotx & transform of a rotation around $X$ axis \\
\hline
roty & transform of a rotation around $Y$ axis \\
\hline
rotz & transform of a rotation around $Z$ axis \\
\hline
trans & transform of a translation \\
\hline
\hline
\end{tabular}
\end{center}
\end{table}

\begin{table}[htbp]
\caption{Quaternion class member functions}
\label{tab:commandsumQuat}
\begin{center}
\begin{tabular}{||l|l||}
\hline
\hline
\multicolumn{2}{||l||}{{\bf Quaternions}} \\
\hline
\hline
+, -, *, /, = & operators on quaternions \\
\hline
conjugate, i  & conjugate (or inverse) of a quaternion \\
\hline
exp, Log, power & exponential, logarithm and power of a quaternion \\
\hline
dot\_prod & dot product of a quaternion \\
\hline
dot, E & quaternion time derivative\\
\hline
unit & make a quaternion a unit quaternion \\
\hline
norm, norm\_sqr & compute the norm and the square norm of a quaternion \\
\hline 
s, v & returns the scalar and the vector of a quaternion \\
\hline
set\_s, set\_v & assign values to the scalar and vector part of a quaternion \\
\hline
R, T  & returns the equivalent rotation matrix ($3 \times 3$ or $4 \times 4$) \\
\hline
\hline
\end{tabular}
\end{center}
\end{table}

\begin{table}[htbp]
\caption{Quaternion non member functions}
\label{tab:commandsumQuatnm}
\begin{center}
\begin{tabular}{||l|l||}
\hline
\hline
\multicolumn{2}{||l||}{{\bf Functions}} \\
\hline
\hline
Omega & returns angular velocity \\
\hline
Slerp  & Spherical Linear Interpolation \\
\hline
Slerp\_prime & Spherical Linear Interpolation derivative \\ 
\hline
Squad & Spherical Cubic Interpolation \\
\hline
Squad\_prime & Spherical Cubic Interpolation derivative \\
\hline
\hline 
\end{tabular}
\end{center}
\end{table}

\begin{table}[htbp]
\caption{Spl\_Quaternion class member function}
\label{tab:commandsum1b}
\begin{center}
\begin{tabular}{||l|l||}
\hline
\hline
\multicolumn{2}{||l||}{{\bf Spl\_Quaternion}} \\
\hline
\hline
quat & interpolate the spline at time $t$ to sets the quaternion $q$. \\
\hline
quat\_w & interpolate the spline at time $t$ to sets the quaternion
$q$ and angular velocity $\omega$. \\
\hline
\hline
\end{tabular}
\end{center}
\end{table}

\begin{table}[htbp]
\caption{Spl\_Cubic class member function}
\label{tab:commandsum1c}
\begin{center}
\begin{tabular}{||l|l||}
\hline
\hline
\multicolumn{2}{||l||}{{\bf Spl\_Cubic}} \\
\hline
\hline
interpolating & interpolate the spline at time $t$. \\
\hline
first\_derivative & interpolate the spline first derivative at time $t$. \\
\hline
second\_derivative & interpolate the spline second derivative at time $t$. \\
\hline
\hline
\end{tabular}
\end{center}
\end{table}

\begin{table}[htbp]
\caption{Spl\_path class member function}
\label{tab:commandsum1d}
\begin{center}
\begin{tabular}{||l|l||}
\hline
\hline
\multicolumn{2}{||l||}{{\bf Spl\_path}} \\
\hline
\hline
p & interpolate the spline at time $t$ to sets the position. \\
\hline
p\_pdot & interpolate the spline at time $t$ to sets position and velocity. \\
\hline
p\_pdot\_pddot & interpolate the spline at time $t$ to sets position,
velocity and acceleration. \\
\hline
\hline
\end{tabular}
\end{center}
\end{table}

\begin{table}[htbp]
\caption{CLIK class member function}
\label{tab:commandsum1e}
\begin{center}
\begin{tabular}{||l|l||}
\hline
\hline
\multicolumn{2}{||l||}{{\bf CLIK}} \\
\hline
\hline
q\_qdot & sets the desired joint position and joint velocity \\
\hline
\end{tabular}
\end{center}
\end{table}

\begin{table}[htbp]
\caption{Computed\_torque\_method class member function}
\label{tab:commandsum1f}
\begin{center}
\begin{tabular}{||l|l||}
\hline
\hline
\multicolumn{2}{||l||}{{\bf Computed\_torque\_method}} \\
\hline
\hline
torque\_cmd & sets the output torque \\
\hline
set\_Kd & sets the derivative error gain \\
\hline
set\_Kp & sets the position error gain \\
\hline
\end{tabular}
\end{center}
\end{table}

\begin{table}[htbp]
\caption{Resolve\_acc class member function}
\label{tab:commandsum1g}
\begin{center}
\begin{tabular}{||l|l||}
\hline
\hline
\multicolumn{2}{||l||}{{\bf Resolve\_acc}} \\
\hline
\hline
torque\_cmd & sets the output torque \\
\hline
set\_Kvp & sets the translational velocity error gain \\
\hline
set\_Kpp & sets the translational position error gain \\
\hline
set\_Kvo & sets the rotational velocity error gain \\
\hline
set\_Kpo & sets the rotational position error gain \\
\hline
\end{tabular}
\end{center}
\end{table}

\begin{table}[htbp]
\caption{Impedance class member function}
\label{tab:commandsum1h}
\begin{center}
\begin{tabular}{||l|l||}
\hline
\hline
\multicolumn{2}{||l||}{{\bf Impedance}} \\
\hline
\hline
control & sets the compliant trajectory \\
\hline
set\_Mp & sets the translational impedance inertia matrix \\
\hline
set\_Dp & sets the translational impedance damping matrix \\
\hline
set\_Kp & sets the translational impedance stiffness matrix \\
\hline
set\_Mo & sets the rotational impedance inertia matrix \\
\hline
set\_Do & sets the rotational impedance damping matrix \\
\hline
set\_Ko & sets the rotational impedance stiffness matrix \\
\hline
\end{tabular}
\end{center}
\end{table}

\begin{table}[htbp]
\caption{IO\_matrix\_file class member functions}
\label{tab:commandsum1i}
\begin{center}
\begin{tabular}{||l|l||}
\hline
\hline
\multicolumn{2}{||l||}{{\bf IO\_matrix\_file}} \\
\hline
\hline
write & create and write data to a file \\
\hline
read & read data from a file  \\
\hline
read\_all & read entire data file at once \\
\hline
\end{tabular}
\end{center}
\end{table}

\begin{table}[htbp]
\caption{Plot2d class member functions}
\label{tab:commandsum1j}
\begin{center}
\begin{tabular}{||l|l||}
\hline
\hline
\multicolumn{2}{||l||}{{\bf Plot2d}} \\
\hline
\hline
addcommand & add a {\tt gnuplot} command the 2d graph \\
\hline
addcurve & add a curve to the 2d graph \\
\hline
dump & dump the content of the graph to {\tt stdout} \\
\hline
gnuplot & plot the graph through a call to \textsf{gnuplot} \\
\hline
settitle & sets graph title  \\
\hline
setxlabel & sets axis X label \\
\hline
setylabel & sets axis Y label \\
\hline
set\_plot2d & ``wrapper'' function for \texttt{Plot2d} \\
\hline
\hline
\end{tabular}
\end{center}
\end{table}

\begin{table}[htbp]
\caption{Plot\_file class member functions}
\label{tab:commandsum1k}
\begin{center}
\begin{tabular}{||l|l||}
\hline
\hline
\multicolumn{2}{||l||}{{\bf Plot\_file}} \\
\hline
\hline
graph & create a graphics from a data file \\
\hline
\hline
\end{tabular}
\end{center}
\end{table}

\begin{table}[htbp]
\caption{Config class member functions}
\label{tab:commandsum1l}
\begin{center}
\begin{tabular}{||l|l||}
\hline
\hline
\multicolumn{2}{||l||}{{\bf Config}} \\
\hline
\hline
read\_conf & read configuration file \\
\hline
select & assign the value of parameter from a section  \\
\hline
add & specify the value of parameter for a section  \\
\hline
\hline
\end{tabular}
\end{center}
\end{table}

\begin{table}[htbp]
\caption{\texttt{Robot} (and \texttt{mRobot}) class member functions}
\label{tab:commandsum2}
\begin{center}
\begin{tabular}{||l|l||}
\hline
\hline
\multicolumn{2}{||l||}{{\bf Joint Variables}} \\
\hline
\hline
get\_q & get the robot joint variables position \\
\hline
get\_qp & get the robot joint variables velocity \\
\hline
get\_qpp & get the robot joint variables acceleration \\
\hline
set\_q & set the robot joint variables position \\
\hline
set\_qp & set the robot joint variables velocity \\
\hline
set\_qpp & set the robot joint variables acceleration \\
\hline
\hline
\multicolumn{2}{||l||}{{\bf Robot Kinematics}} \\
\hline
\hline
inv\_kin & inverse kinematics \\
inv\_kin\_rhino & Rhino inverse kinematics \\
inv\_kin\_puma & Puma inverse kinematics \\
\hline
jacobian & robot Jacobian \\
jacobian\_dot & robot Jacobian derivative \\
jacobian\_DLS\_inv & robot Jacobian DLS inverse\\
\hline
kine, kine\_pd & forward kinematics \\
\hline
dTdqi & partial derivative of forward kinematics \\
\hline
\hline
\multicolumn{2}{||l||}{{\bf Robot Dynamics}} \\
\hline
\hline
acceleration & forward dynamics \\
\hline
inertia & robot inertia matrix \\
\hline
torque & inverse dynamics \\
\hline
torque\_novelocity & inverse dynamics without velocity and gravity \\
\hline
G & gravity effects \\
\hline
C & Coriolis and centrifugal effects \\
\hline
\hline
\multicolumn{2}{||l||}{{\bf Robot Linearized Dynamics}} \\
\hline
\hline
delta\_torque & 
$\delta \mbold{\tau} = \mbold{D(q)} \delta \ddot{\mbold{q}} 
+ \mbold{S_1(q,}\dot{\mbold{q}}) \delta \dot{\mbold{q}}
+ \mbold{S_2(q,}\dot{\mbold{q}},\ddot{\mbold{q}}) \delta \mbold{q}
$ \\
\hline
dq\_torque & 
$ \mbold{S_2(q,}\dot{\mbold{q}},\ddot{\mbold{q}}) \delta \mbold{q}
$ \\
\hline
dqp\_torque & 
$ \mbold{S_1(q,}\dot{\mbold{q}}) \delta \dot{\mbold{q}}
$ \\
\hline
dtau\_dq & 
$\dfdx{\mbold{\tau}}{\mbold{q}} = \mbold{S_2(q,}\dot{\mbold{q}},\ddot{\mbold{q}})
$ \\
\hline
dtau\_dqp & 
$\dfdx{\mbold{\tau}}{\dot{\mbold{q}}} = \mbold{S_1(q,}\dot{\mbold{q}}) 
$ \\
\hline
\hline
\end{tabular}
\end{center}
\end{table}

\begin{table}[htbp]
\caption{Miscellaneous}
\label{tab:commandsum3}
\begin{center}
\begin{tabular}{||l|l||}
\hline
\hline
\multicolumn{2}{||l||}{{\bf Miscellaneous}} \\
\hline
\hline
odeint & adaptive step size Runge-Kutta integrator \\
\hline
Runge\_Kutta4 & fixed step size $4^{th}$ order Runge-Kutta integrator \\
\hline
Integ\_Trap & trapezoidal integration \\
\hline
pinv & matrix pseudo inverse \\
\hline
vec\_dot\_prod & vector dot product \\
\hline
vec\_x\_prod & vector cross product \\
\hline
x\_prod\_matrix & cross product matrix\\
\hline
perturb\_robot & perturb robot parameters \\
\hline
\hline
\end{tabular}
\end{center}
\end{table}


\chapter{Reporting bugs, contributions and comments}

I intend to support this library. By this, I mean that bugs will be fixed as fast as
time allows me and that new functionalities will be introduced in future releases.
If you find a bug or think some part of the documentation could be improved, 
let me know and I will try to include the corrections in the next release.
Comments regarding the documentation will not be treated as fast as bug reports. 
I will not, however, help users with problems related to assignments and homework.
You can use your Web browser to send comments or bug report with the URL: 

{\tt \htmladdnormallink{http://sourceforge.net/projects/roboop/}{http://sourceforge.net/projects/roboop/}}.

\section{Reporting bugs}

When reporting bugs, please send the following information (see the file {\tt bugs.txt}):
\begin{verbatim}
VERSION OF THE PACKAGE (see the readme.txt file):

OS:

COMPILER:

DESCRIPTION OF THE BUG:

SAMPLE CODE THAT MAKE THE BUG APPARENT:

\end{verbatim}
or use the URL: {\tt \htmladdnormallink{http://sourceforge.net/projects/roboop/}{http://sourceforge.net/projects/roboop/}}.


\section{Making a contribution to the package}

If you have written some code you think might be useful for other users of the package,
I will be happy to integrate it in future releases. Makefiles for compilers not included
in this distribution would be greatly appreciated. Contact me for more details: 
{\tt \htmladdnormallink{richard.gourdeau@polymtl.ca}{mailto:richard.gourdeau@polymtl.ca}}.

\section{Citing the package}

If you are using the \textsf{ROBOOP} package, please let me know. If you want to cite this package
in some of your work, please use \cite{Gourdeau97} or the following {\rmfamily B\kern-.05em%
    \textsc{i\kern-.025em b}\kern-.08em%
    T\kern-.1667em\lower.7ex\hbox{E}\kern-.125emX} entry:
\begin{verbatim}
@ARTICLE{Gourdeau97,
  author = {Richard Gourdeau},
  month = sep,
  year = 1997,
  title = {Object Oriented Programming for Robotic
           Manipulators Simulation},
  journal = {IEEE Robotics and Automation Magazine},
  volume = 4,
  number = 3,
  pages = {21--29}
}
\end{verbatim}

\chapter{Credits and acknowledgments}

I would like to thank Robert Davies for making his \textsf{NEWMAT11}
library available.

The hardware and software used to develop the initial releases of this
package were funded through NSERC grants OGP0138478 and EQP0172766.

I would like to thank Etienne Lachance for his contributions since the
1.13 release and Samuel Belanger for the initial version of the
Stewart class.

\chapter{Future developments}

In future releases, we hope to include the following:
\begin{itemize}
\item functions for basic control laws (sliding modes, etc);
\item make files for other compilers.
\end{itemize}


\bibliographystyle{ieee}
\bibliography{ooprsim}

\appendix

% $Id: rne.tex,v 1.6 2007/10/18 16:20:52 gourdeau Exp $

\chapter{Recursive Newton-Euler algorithms, DH notation}
\label{sec:rnea}

In order to apply the RNE as presented in \cite{Murray86}, 
let us define the following variables 
(referenced in the $i^{th}$ coordinate frame if applicable):
\begin{itemize}
\item $\sigma_i$ is the joint type; $\sigma_i = 1$ for a revolute
  joint and $\sigma_i = 0$ for a prismatic joint;
\item $\mbold{p}_i = \matr{ccc}{ a_i & d_i \sin \alpha_i & d_i \cos \alpha_i}^T$ is the
position of the $i^{th}$ with respect to the $i-1^{th}$ frame;
\item $\mbold{z}_0 = \matr{ccc}{0 & 0 & 1}^T $
\end{itemize}


\section{Recursive Newton-Euler formulation}
\label{sec:rne}

\begin{itemize}
\item Forward Iterations for $i=1, 2, \ldots, n$. 

Initialize: $\omega_0 = \dot{\omega}_0 = 0$ and $\dot{\mbold{v}}_0 = - \mbold{g}$.
\begin{eqnarray}
\omega_i & = & \mbold{R}_i^T [\omega_{i-1} + \sigma_i \mbold{z}_0 \dot{\theta}_i ] \\
\dot{\omega}_i & = & \mbold{R}_i^T  \{ \dot{\omega}_{i-1} + 
\sigma_i [\mbold{z}_0 \ddot{\theta}_i + \omega_{i-1} \times (\mbold{z}_0 \dot{\theta}_i )] \} \\
\dot{\mbold{v}}_i & = & \mbold{R}_i^T  \{ \dot{\mbold{v}}_{i-1} + 
(1 -\sigma_i) [\mbold{z}_0 \ddot{d}_i + 2 \omega_{i-1} \times (\mbold{z}_0 \dot{d}_i )] \} 
\nonumber \\ 
& & + \dot{\omega}_i \times \mbold{p}_i + \omega_i \times ( \omega_i \times \mbold{p}_i)
\end{eqnarray}

\item Backward Iterations for $i=n, n-1, \ldots, 1$. 

Initialize: $\mbold{f}_{n+1} = \mbold{n}_{n+1} = 0$.
\begin{eqnarray}
\dot{\mbold{v}}_{ci} & = & \dot{\mbold{v}}_i + \dot{\omega}_i \times \mbold{r}_i 
+ \omega_i \times (\omega_i \times \mbold{r}_i) \\
\mbold{F}_i & = & m_i \dot{\mbold{v}}_{ci} \\
\mbold{N}_i & = & \mbold{I}_{ci} \dot{\omega}_i + \omega_i \times (\mbold{I}_{ci} \omega_i) \\
\mbold{f}_i & = & \mbold{R}_{i+1} [ \mbold{f}_{i+1} ]  + \mbold{F}_{i}  \\
\mbold{n}_i & = & \mbold{R}_{i+1} [ \mbold{n}_{i+1} ]  + \mbold{p}_{i} \times \mbold{f}_{i} 
+ \mbold{N}_{i} + \mbold{r}_{i} \times \mbold{F}_{i} \\
\tau_i & = & \sigma_i \mbold{n}_i^T (\mbold{R}_i^T \mbold{z}_0) 
+ (1 -\sigma_i) \mbold{f}_i^T (\mbold{R}_i^T \mbold{z}_0)
\end{eqnarray}
\end{itemize}


\section{Recursive linearized Newton-Euler formulation}
\label{sec:rlne}

With
\begin{eqnarray}
\mbold{p}_{di} & = & \dfdx{\mbold{p}_i}{d_i} = \matr{ccc}{ 0 & \sin \alpha_i & \cos \alpha_i}^T \\
\mbold{Q} & = & \matr{ccc}{
0 & -1 & 0  \\
1 & 0 & 0  \\
0 & 0 & 0 }
\end{eqnarray}
one can use the following
\begin{itemize}
\item Forward Iterations for $i=1, 2, \ldots, n$. 

Initialize: $\delta \omega_0 = \delta \dot{\omega}_0 = \delta \dot{\mbold{v}}_0 = 0$.
\begin{eqnarray}
\delta \omega_i & = & 
\mbold{R}_i^T \{\delta \omega_{i-1} + \sigma_i [ \mbold{z}_0 \delta \dot{\theta}_i 
- \mbold{Q}(\omega_{i-1} + \dot{\theta}_i ) \delta \theta_i ] \} \\
\delta \dot{\omega}_i & = & 
\mbold{R}_i^T  \{ \delta \dot{\omega}_{i-1} + 
\sigma_i [\mbold{z}_0 \delta \ddot{\theta}_i + \delta \omega_{i-1} \times (\mbold{z}_0 \dot{\theta}_i )
+ \omega_{i-1} \times (\mbold{z}_0 \delta \dot{\theta}_i )]  \nonumber \\
& & - \sigma_i \mbold{Q} [ \omega_{i-1} + \mbold{z}_0 \ddot{\theta}_i 
+ \omega_{i-1} \times (\mbold{z}_0 \dot{\theta}_i )] \delta \theta_i \} \\
\delta \dot{\mbold{v}}_i & = & 
\mbold{R}_i^T  \{ \delta \dot{\mbold{v}}_{i-1} 
- \sigma_i \mbold{Q} \dot{\mbold{v}}_{i-1} \delta \theta_i \nonumber \\
& & + (1 -\sigma_i) [\mbold{z}_0 \delta \ddot{d}_i 
+ 2 \delta \omega_{i-1} \times (\mbold{z}_0 \dot{d}_i ) 
+ 2 \omega_{i-1} \times (\mbold{z}_0 \delta \dot{d}_i )] \} 
\nonumber \\ 
& & + \delta \dot{\omega}_i \times \mbold{p}_i 
+ \delta \omega_i \times ( \omega_i \times \mbold{p}_i) 
+ \omega_i \times ( \delta \omega_i \times \mbold{p}_i) \nonumber \\
& & + (1 - \sigma_i) (\dot{\omega}_i \times \mbold{p}_{di} 
+ \omega_i \times ( \omega_i \times \mbold{p}_{di}) ) \delta d_i
\end{eqnarray}

\item Backward Iterations for $i=n, n-1, \ldots, 1$. 

Initialize: $\delta \mbold{f}_{n+1} = \delta \mbold{n}_{n+1} = 0$.
\begin{eqnarray}
\delta \dot{\mbold{v}}_{ci} & = & 
\delta \dot{\mbold{v}}_i + \delta \dot{\omega}_i \times \mbold{r}_i 
+ \delta \omega_i \times (\omega_i \times \mbold{r}_i) 
+ \omega_i \times (\delta \omega_i \times \mbold{r}_i) \\
\delta \mbold{F}_i & = & m_i \delta \dot{\mbold{v}}_{ci} \\
\delta \mbold{N}_i & = & \mbold{I}_{ci} \delta \dot{\omega}_i 
+ \delta \omega_i \times (\mbold{I}_{ci} \omega_i) 
+ \omega_i \times (\mbold{I}_{ci} \delta \omega_i) \\
\delta \mbold{f}_i & = & \mbold{R}_{i+1} [ \delta \mbold{f}_{i+1} ]  
+ \delta \mbold{F}_{i} + \sigma_{i+1} \mbold{Q} \mbold{R}_{i+1} [ \mbold{f}_{i+1} ] \delta \theta_{i+1} \\
\delta \mbold{n}_i & = & \mbold{R}_{i+1} [ \delta \mbold{n}_{i+1} ]  
+ \delta \mbold{N}_{i} + \mbold{p}_{i} \times \delta \mbold{f}_{i} 
+ \mbold{r}_{i} \times \delta \mbold{F}_{i} \nonumber \\
& & + (1 - \sigma_i) (\mbold{p}_{di} \times \mbold{f}_{i}) \delta d_i 
+ \sigma_{i+1} \mbold{Q} \mbold{R}_{i+1} [ \mbold{n}_{i+1} ] \delta \theta_{i+1} \\
\delta \tau_i & = & \sigma_i [ \delta \mbold{n}_i^T (\mbold{R}_i^T \mbold{z}_0) 
- \mbold{n}_i^T (\mbold{R}_i^T \mbold{Q} \mbold{z}_0) \delta \theta_i] \nonumber \\
& & + (1 -\sigma_i) [ \delta \mbold{f}_i^T (\mbold{R}_i^T \mbold{z}_0) ]
\end{eqnarray}
\end{itemize}

%---------------------------------------------------------

\chapter{Recursive Newton-Euler algorithms, modified DH notation}

In order to apply the RNE, let us define the following variables 
(referenced in the $i^{th}$ coordinate frame if applicable):
\begin{itemize}
\item $\sigma_i$ is the joint type; $\sigma_i = 1$ for a revolute
  joint and $\sigma_i = 0$ for a prismatic joint;
\item $\mbold{p}_i = \matr{ccc}{a_{i-1} & -d_i sin \alpha_{i-1} &
d_i cos \alpha_{i-1}}^T$ is the
position of the $i^{th}$ with respect to the $i-1^{th}$ frame;
\item $\mbold{z}_0 = \matr{ccc}{0 & 0 & 1}^T $
\end{itemize}


\section{Recursive Newton-Euler formulation}

\begin{itemize}
\item Forward Iterations for $i=1, 2, \ldots, n$. 

Initialize: $\omega_0 = \dot{\omega}_0 = 0$ and $\dot{\mbold{v}}_0 = - \mbold{g}$.
\begin{eqnarray}
\omega_i & = & \mbold{R}_i^T\omega_{i-1} + \sigma_i \mbold{z}_0\dot{\theta_i} \\
\dot{\omega}_i & = & \mbold{R}_i^T\dot{\omega}_{i-1} + \sigma_i
 \mbold{R}_i^T\omega_{i-1}\times \mbold{z}_0 \dot{\theta}_i + \sigma_i\mbold{z}_0\ddot{\theta}_i \\
\dot{\mbold{v}}_i & = &\ \mbold{R}_i^T(\dot{\omega}_{i-1}\times \mbold{p}_i\ +\
\omega_{i-1}\times(\omega_{i-1}\times \mbold{p}_i) + \dot{\mbold{v}}_{i-1})\  \\ \nonumber
            & & + (1\ -\ \sigma_i)(2\omega_i\times \mbold{z}_0\dot{d}_i\ + \mbold{z}_0\ddot{d}_i)
\end{eqnarray}

\item Backward Iterations for $i=n, n-1, \ldots, 1$. 

Initialize: $\mbold{f}_{n+1} = \mbold{n}_{n+1} = 0$.
\begin{eqnarray}
\dot{\mbold{v}}_{ci} & = & \dot{\omega}_i\times \mbold{r}_i + 
\omega_i\times(\omega_i\times \mbold{r}_i) + \dot{\mbold{v}}_i \\
\mbold{F}_i & = & m_i \dot{\mbold{v}}_{ci} \\
\mbold{N}_i & = & \mbold{I}_{ci}\ddot{\omega}_i\ + \omega_i \times \mbold{I}_{ci}\omega_i \\
\mbold{f}_i & = & \mbold{R}_{i+1}\mbold{f}_{i+1} + \mbold{F}_i \\
\mbold{n}_i & = & \mbold{N}_i + \mbold{R}_{i+1} \mbold{n}_{i+1} +
\mbold{r}_i \times \mbold{F}_i + \mbold{p}_{i+1}\times \mbold{R}_{i+1}\mbold{f}_{i+1} \\
\tau_i & = & \sigma_i \mbold{n}_i \mbold{z}_) + (1 - \sigma_i)
\mbold{f}_i^T \mbold{z}_0
\end{eqnarray}
\end{itemize}


\section{Recursive linearized Newton-Euler formulation}


With
\begin{eqnarray}
\mbold{p}_{di} & = & \dfdx{\mbold{p}_i}{d_i} = \matr{ccc}{ 0 & -\sin\alpha_{i-1} & \cos\alpha_{i-1}}^T \\
\mbold{Q} & = & \matr{ccc}{
0 & -1 & 0  \\
1 & 0 & 0  \\
0 & 0 & 0 }
\end{eqnarray}
one can use the following
\begin{itemize}
\item Forward Iterations for $i=1, 2, \ldots, n$. 

Initialize: $\delta \omega_0 = \delta \dot{\omega}_0 = \delta \dot{\mbold{v}}_0 = 0$.
\begin{eqnarray}
\delta \omega_i & = & \mbold{R}_i^T \delta \omega_{i-1} + \sigma_i
(\mbold{z}_0 \delta \dot{\theta}_i\ -\ \mbold{QR}_i^T \omega_i \delta \theta_i) \\
\delta \dot{\omega}_i  & = & \mbold{R}_i^T \delta \dot{\mbold{w}}_{i-1} +
\sigma_i [\mbold{R}_i^T \delta \omega_{i-1} \times \mbold{z}_0
\dot{\theta}_i \\ \nonumber
& & + \mbold{R}_i^T \omega_{i-1} \times \mbold{z}_0 \delta \dot{\theta}_i  
+ \mbold{z}_0 \ddot{\theta}_i \\ \nonumber
& & - (\mbold{QR}_i^T \dot{\omega}_{i-1} + \mbold{QR}_i^T \omega_{i-1}
\times \omega{z}_0 \dot{\theta}_i) \delta \theta_i ] \\
\delta \dot{\mbold{v}}_i & = & \mbold{R}_i^T\Big(\delta
\dot{\omega}_{i-1} \times \mbold{p}_i + \delta \omega_{i-1}
\times(\omega_{i-1} \times \mbold{p}_i) \\ \nonumber
& & + \omega_{i-1} \times( \delta \omega_{i-1} \times \mbold{p}_i) +
\delta \dot{\mbold{v}}_i\Big) \\ \nonumber
& & + (1-\sigma_i)\Big(2\delta \omega_i \times \mbold{z}_0\dot{\mbold{d}}_i
+ 2\omega_i \times \mbold{z}_0 \delta \dot{\mbold{d}}_i + \mbold{z}_0
\delta \ddot{\mbold{d}}_i\Big) \\ \nonumber
& & - \sigma_i\mbold{QR}_i^T \Big(\dot{\omega}_{i-1} \times
\mbold{p}_i + \omega_{i-1} \times(w_{i-1}\times p_i)
+ \dot{\mbold{v}}_i\Big) \delta \theta_i \\ \nonumber
& & + (1-\sigma_i) \mbold{R}_i^T \Big(\dot{\omega}_{i-1} \times
\mbold{p}_{di} + \omega_{i-1} \times(\omega_{i-1}\times
\mbold{p}_{di})\Big) \delta \mbold{d}_i
\end{eqnarray}

\item Backward Iterations for $i=n, n-1, \ldots, 1$. 

Initialize: $\delta \mbold{f}_{n+1} = \delta \mbold{n}_{n+1} = 0$.
\begin{eqnarray}
\delta \dot{\mbold{v}}_{ci} & = & \delta \dot{\mbold{v}}_i + \delta \dot{\omega}_i\times
\mbold{r}_i + \delta \omega_i \times(\omega_i \times \mbold{r}_i) \\ \nonumber
& & + \omega_i \times ( \delta \omega_i \times \mbold{r}_i) \\
\delta \mbold{F}_i & = & m_i \delta \dot{\mbold{v}}_{ci} \\
\delta \mbold{N}_i & = & \mbold{I}_{ci} \delta \dot{\omega}_i + \delta
\omega_i \times (\mbold{I}_{ci}\omega_i) + \omega_i \times
(\mbold{I}_{ci} \delta \omega_i) \\
\delta \mbold{f}_i & = & \mbold{R}_{i+1} \delta \mbold{f}_{i+1} +
\delta \mbold{F}_i + \sigma_{i+1} \mbold{R}_{i+1} \mbold{Qf}_{i+1} \delta \theta_{i+1} \\
\delta \mbold{n}_i & = & \delta \mbold{N}_i + \mbold{R}_{i+1} \delta
\mbold{n}_{i+1} + \mbold{r}_i\times \delta \mbold{F}_i \\ \nonumber
& & + \mbold{p}_{i+1} \times \mbold{R}_{i+1} \delta \mbold{f}_{i+1} \\ \nonumber  
& & + \sigma_{i+1} \Big( \mbold{R}_{i+1} \mbold{Qn}_{i+1}
+ \mbold{p}_{i+1} \times \mbold{R}_{i+1} \mbold{Qf}_{i+1} \Big) \delta
\theta_{i+1} \\ \nonumber
& & + (1-\sigma_{i+1} ) \mbold{p}_{di+1} \mbold{p}_{di+1} \times \mbold{R}_{i+1}
\mbold{f}_{i+1} \delta d_{i+1}  \\
\delta \mbold{\tau}_i & = & \sigma \delta\mbold{n}_i^T \mbold{z}_0 +
(1 - \sigma_i) \delta\mbold{f}_i^T \mbold{z}_0
\end{eqnarray}
\end{itemize}















\chapter{GNU Lesser General Public License}
\label{chap:gnulgpl}

Content of the file {\tt GNUlgpl.txt}.

\subsubsection*{GNU LESSER GENERAL PUBLIC LICENSE}

Version 2.1, February 1999

\noindent Copyright (C) 1991, 1999 Free Software Foundation, Inc. \\
59 Temple Place, Suite 330, Boston, MA  02111-1307  USA \\
Everyone is permitted to copy and distribute verbatim copies of this
license document, but changing it is not allowed.

\noindent [This is the first released version of the Lesser GPL. It also 
counts as the successor of the GNU Library Public License, version 2, 
hence the version number 2.1.]

\subsubsection*{Preamble}

The licenses for most software are designed to take away your freedom
to share and change it.  By contrast, the GNU General Public Licenses
are intended to guarantee your freedom to share and change free
software---to make sure the software is free for all its users.

This license, the Lesser General Public License, applies to some
specially designated software packages---typically libraries---of the
Free Software Foundation and other authors who decide to use it.  You
can use it too, but we suggest you first think carefully about whether
this license or the ordinary General Public License is the better
strategy to use in any particular case, based on the explanations
below.

When we speak of free software, we are referring to freedom of use,
not price.  Our General Public Licenses are designed to make sure that
you have the freedom to distribute copies of free software (and charge
for this service if you wish); that you receive source code or can get
it if you want it; that you can change the software and use pieces of
it in new free programs; and that you are informed that you can do
these things.

To protect your rights, we need to make restrictions that forbid
distributors to deny you these rights or to ask you to surrender these
rights.  These restrictions translate to certain responsibilities for
you if you distribute copies of the library or if you modify it.

For example, if you distribute copies of the library, whether gratis
or for a fee, you must give the recipients all the rights that we gave
you.  You must make sure that they, too, receive or can get the source
code.  If you link other code with the library, you must provide
complete object files to the recipients, so that they can relink them
with the library after making changes to the library and recompiling
it.  And you must show them these terms so they know their rights.

We protect your rights with a two-step method: (1) we copyright the
library, and (2) we offer you this license, which gives you legal
permission to copy, distribute and/or modify the library.

To protect each distributor, we want to make it very clear that there
is no warranty for the free library.  Also, if the library is modified
by someone else and passed on, the recipients should know that what
they have is not the original version, so that the original author's
reputation will not be affected by problems that might be introduced
by others.


Finally, software patents pose a constant threat to the existence of
any free program.  We wish to make sure that a company cannot
effectively restrict the users of a free program by obtaining a
restrictive license from a patent holder.  Therefore, we insist that
any patent license obtained for a version of the library must be
consistent with the full freedom of use specified in this license.

  Most GNU software, including some libraries, is covered by the
ordinary GNU General Public License.  This license, the GNU Lesser
General Public License, applies to certain designated libraries, and
is quite different from the ordinary General Public License.  We use
this license for certain libraries in order to permit linking those
libraries into non-free programs.

When a program is linked with a library, whether statically or using a
shared library, the combination of the two is legally speaking a
combined work, a derivative of the original library.  The ordinary
General Public License therefore permits such linking only if the
entire combination fits its criteria of freedom.  The Lesser General
Public License permits more lax criteria for linking other code with
the library.

We call this license the "Lesser" General Public License because it
does Less to protect the user's freedom than the ordinary General
Public License.  It also provides other free software developers Less
of an advantage over competing non-free programs.  These disadvantages
are the reason we use the ordinary General Public License for many
libraries.  However, the Lesser license provides advantages in certain
special circumstances.

For example, on rare occasions, there may be a special need to
encourage the widest possible use of a certain library, so that it
becomes a de-facto standard.  To achieve this, non-free programs must
be allowed to use the library.  A more frequent case is that a free
library does the same job as widely used non-free libraries.  In this
case, there is little to gain by limiting the free library to free
software only, so we use the Lesser General Public License.

In other cases, permission to use a particular library in non-free
programs enables a greater number of people to use a large body of
free software.  For example, permission to use the GNU C Library in
non-free programs enables many more people to use the whole GNU
operating system, as well as its variant, the GNU/Linux operating
system.

Although the Lesser General Public License is Less protective of the
users' freedom, it does ensure that the user of a program that is
linked with the Library has the freedom and the wherewithal to run
that program using a modified version of the Library.

The precise terms and conditions for copying, distribution and
modification follow.  Pay close attention to the difference between a
"work based on the library" and a "work that uses the library".  The
former contains code derived from the library, whereas the latter must
be combined with the library in order to run.

\subsubsection*{GNU LESSER GENERAL PUBLIC LICENSE
  TERMS AND CONDITIONS FOR COPYING, DISTRIBUTION AND MODIFICATION}

{\small

  \begin{itemize}
  \item [0.] This License Agreement applies to any software library or
    other program which contains a notice placed by the copyright
    holder or other authorized party saying it may be distributed
    under the terms of this Lesser General Public License (also called
    "this License").  Each licensee is addressed as "you".
    
    A "library" means a collection of software functions and/or data
    prepared so as to be conveniently linked with application programs
    (which use some of those functions and data) to form executables.
    
    The "Library", below, refers to any such software library or work
    which has been distributed under these terms.  A "work based on
    the Library" means either the Library or any derivative work under
    copyright law: that is to say, a work containing the Library or a
    portion of it, either verbatim or with modifications and/or
    translated straightforwardly into another language.  (Hereinafter,
    translation is included without limitation in the term
    "modification".)
    
    "Source code" for a work means the preferred form of the work for
    making modifications to it.  For a library, complete source code
    means all the source code for all modules it contains, plus any
    associated interface definition files, plus the scripts used to
    control compilation and installation of the library.
    
    Activities other than copying, distribution and modification are
    not covered by this License; they are outside its scope.  The act
    of running a program using the Library is not restricted, and
    output from such a program is covered only if its contents
    constitute a work based on the Library (independent of the use of
    the Library in a tool for writing it).  Whether that is true
    depends on what the Library does and what the program that uses
    the Library does.
  
  \item [1.] You may copy and distribute verbatim copies of the
    Library's complete source code as you receive it, in any medium,
    provided that you conspicuously and appropriately publish on each
    copy an appropriate copyright notice and disclaimer of warranty;
    keep intact all the notices that refer to this License and to the
    absence of any warranty; and distribute a copy of this License
    along with the Library.
    
    You may charge a fee for the physical act of transferring a copy,
    and you may at your option offer warranty protection in exchange
    for a fee. 
    
  \item [2.] You may modify your copy or copies of the Library or any
    portion of it, thus forming a work based on the Library, and copy
    and distribute such modifications or work under the terms of
    Section 1 above, provided that you also meet all of these
    conditions:
    
    \begin{itemize}
    \item [a)] The modified work must itself be a software library.
    
    \item [b)] You must cause the files modified to carry prominent
      notices stating that you changed the files and the date of any
      change.
    
    \item [c)] You must cause the whole of the work to be licensed at
      no charge to all third parties under the terms of this License.
    
    \item [d)] If a facility in the modified Library refers to a
      function or a table of data to be supplied by an application
      program that uses the facility, other than as an argument passed
      when the facility is invoked, then you must make a good faith
      effort to ensure that, in the event an application does not
      supply such function or table, the facility still operates, and
      performs whatever part of its purpose remains meaningful.
    
      (For example, a function in a library to compute square roots
      has a purpose that is entirely well-defined independent of the
      application.  Therefore, Subsection 2d requires that any
      application-supplied function or table used by this function
      must be optional: if the application does not supply it, the
      square root function must still compute square roots.)
    
  \end{itemize}
  
  These requirements apply to the modified work as a whole.  If
  identifiable sections of that work are not derived from the Library,
  and can be reasonably considered independent and separate works in
  themselves, then this License, and its terms, do not apply to those
  sections when you distribute them as separate works.  But when you
  distribute the same sections as part of a whole which is a work
  based on the Library, the distribution of the whole must be on the
  terms of this License, whose permissions for other licensees extend
  to the entire whole, and thus to each and every part regardless of
  who wrote it.
  
  Thus, it is not the intent of this section to claim rights or
  contest your rights to work written entirely by you; rather, the
  intent is to exercise the right to control the distribution of
  derivative or collective works based on the Library.
  
  In addition, mere aggregation of another work not based on the
  Library with the Library (or with a work based on the Library) on a
  volume of a storage or distribution medium does not bring the other
  work under the scope of this License.
  
\item [3.] You may opt to apply the terms of the ordinary GNU General
  Public License instead of this License to a given copy of the
  Library.  To do this, you must alter all the notices that refer to
  this License, so that they refer to the ordinary GNU General Public
  License, version 2, instead of to this License.  (If a newer version
  than version 2 of the ordinary GNU General Public License has
  appeared, then you can specify that version instead if you wish.)
  Do not make any other change in these notices.   Once this change
  is made in a given copy, it is irreversible for that copy, so the
  ordinary GNU General Public License applies to all subsequent copies
  and derivative works made from that copy.
  
  This option is useful when you wish to copy part of the code of the
  Library into a program that is not a library.
  
\item [4.] You may copy and distribute the Library (or a portion or
  derivative of it, under Section 2) in object code or executable form
  under the terms of Sections 1 and 2 above provided that you
  accompany it with the complete corresponding machine-readable source
  code, which must be distributed under the terms of Sections 1 and 2
  above on a medium customarily used for software interchange.
    
  If distribution of object code is made by offering access to copy
  from a designated place, then offering equivalent access to copy the
  source code from the same place satisfies the requirement to
  distribute the source code, even though third parties are not
  compelled to copy the source along with the object code.
  
\item [5.] A program that contains no derivative of any portion of the
  Library, but is designed to work with the Library by being compiled
  or linked with it, is called a "work that uses the Library".  Such a
  work, in isolation, is not a derivative work of the Library, and
  therefore falls outside the scope of this License.
  
  However, linking a "work that uses the Library" with the Library
  creates an executable that is a derivative of the Library (because
  it contains portions of the Library), rather than a "work that uses
  the library".  The executable is therefore covered by this License.
  Section 6 states terms for distribution of such executables.
  
  When a "work that uses the Library" uses material from a header file
  that is part of the Library, the object code for the work may be a
  derivative work of the Library even though the source code is not.
  Whether this is true is especially significant if the work can be
  linked without the Library, or if the work is itself a library.  The
  threshold for this to be true is not precisely defined by law.
  
  If such an object file uses only numerical parameters, data
  structure layouts and accessors, and small macros and small inline
  functions (ten lines or less in length), then the use of the object
  file is unrestricted, regardless of whether it is legally a
  derivative work.  (Executables containing this object code plus
  portions of the Library will still fall under Section 6.)
  
  Otherwise, if the work is a derivative of the Library, you may
  distribute the object code for the work under the terms of Section
  6.  Any executables containing that work also fall under Section 6,
  whether or not they are linked directly with the Library itself.
  
\item [6.] As an exception to the Sections above, you may also combine
  or link a "work that uses the Library" with the Library to produce a
  work containing portions of the Library, and distribute that work
  under terms of your choice, provided that the terms permit
  modification of the work for the customer's own use and reverse
  engineering for debugging such modifications.
    
  You must give prominent notice with each copy of the work that the
  Library is used in it and that the Library and its use are covered
  by this License.  You must supply a copy of this License.  If the
  work during execution displays copyright notices, you must include
  the copyright notice for the Library among them, as well as a
  reference directing the user to the copy of this License.  Also, you
  must do one of these things:
    
  \begin{itemize}
  \item [a)] Accompany the work with the complete corresponding
    machine-readable source code for the Library including whatever
    changes were used in the work (which must be distributed under
    Sections 1 and 2 above); and, if the work is an executable linked
    with the Library, with the complete machine-readable "work that
    uses the Library", as object code and/or source code, so that the
    user can modify the Library and then relink to produce a modified
    executable containing the modified Library.  (It is understood
    that the user who changes the contents of definitions files in the
    Library will not necessarily be able to recompile the application
    to use the modified definitions.)
    
  \item [b)] Use a suitable shared library mechanism for linking with
    the Library.  A suitable mechanism is one that (1) uses at run
    time a copy of the library already present on the user's computer
    system, rather than copying library functions into the executable,
    and (2) will operate properly with a modified version of the
    library, if the user installs one, as long as the modified version
    is interface-compatible with the version that the work was made
    with.
    
  \item [c)] Accompany the work with a written offer, valid for at least
    three years, to give the same user the materials specified in
    Subsection 6a, above, for a charge no more than the cost of
    performing this distribution.
    
  \item [d)] If distribution of the work is made by offering access to
    copy from a designated place, offer equivalent access to copy the
    above specified materials from the same place.
    
  \item [e)] Verify that the user has already received a copy of these
    materials or that you have already sent this user a copy.
  \end{itemize}
  
  For an executable, the required form of the "work that uses the
  Library" must include any data and utility programs needed for
  reproducing the executable from it.  However, as a special
  exception, the materials to be distributed need not include anything
  that is normally distributed (in either source or binary form) with
  the major components (compiler, kernel, and so on) of the operating
  system on which the executable runs, unless that component itself
  accompanies the executable.
  
  It may happen that this requirement contradicts the license
  restrictions of other proprietary libraries that do not normally
  accompany the operating system.  Such a contradiction means you
  cannot use both them and the Library together in an executable that
  you distribute.
  
\item [7.] You may place library facilities that are a work based on
  the Library side-by-side in a single library together with other
  library facilities not covered by this License, and distribute such
  a combined library, provided that the separate distribution of the
  work based on the Library and of the other library facilities is
  otherwise permitted, and provided that you do these two things:
    
  \begin{itemize}
  \item [a)] Accompany the combined library with a copy of the same
    work based on the Library, uncombined with any other library
    facilities.  This must be distributed under the terms of the
    Sections above.
    
  \item [b)] Give prominent notice with the combined library of the fact
    that part of it is a work based on the Library, and explaining
    where to find the accompanying uncombined form of the same work.
  \end{itemize}
  
\item [8.] You may not copy, modify, sublicense, link with, or
  distribute the Library except as expressly provided under this
  License.  Any attempt otherwise to copy, modify, sublicense, link
  with, or distribute the Library is void, and will automatically
  terminate your rights under this License.  However, parties who have
  received copies, or rights, from you under this License will not
  have their licenses terminated so long as such parties remain in
  full compliance.
    
\item [9.] You are not required to accept this License, since you have
  not signed it.  However, nothing else grants you permission to
  modify or distribute the Library or its derivative works.  These
  actions are prohibited by law if you do not accept this License.
  Therefore, by modifying or distributing the Library (or any work
  based on the Library), you indicate your acceptance of this License
  to do so, and all its terms and conditions for copying, distributing
  or modifying the Library or works based on it.
    
\item [10.] Each time you redistribute the Library (or any work based
  on the Library), the recipient automatically receives a license from
  the original licensor to copy, distribute, link with or modify the
  Library subject to these terms and conditions.  You may not impose
  any further restrictions on the recipients' exercise of the rights
  granted herein.  You are not responsible for enforcing compliance by
  third parties with this License.   11. If, as a consequence of a
  court judgment or allegation of patent infringement or for any other
  reason (not limited to patent issues), conditions are imposed on you
  (whether by court order, agreement or otherwise) that contradict the
  conditions of this License, they do not excuse you from the
  conditions of this License.  If you cannot distribute so as to
  satisfy simultaneously your obligations under this License and any
  other pertinent obligations, then as a consequence you may not
  distribute the Library at all.  For example, if a patent license
  would not permit royalty-free redistribution of the Library by all
  those who receive copies directly or indirectly through you, then
  the only way you could satisfy both it and this License would be to
  refrain entirely from distribution of the Library.
    
  If any portion of this section is held invalid or unenforceable
  under any particular circumstance, the balance of the section is
  intended to apply, and the section as a whole is intended to apply
  in other circumstances.
    
  It is not the purpose of this section to induce you to infringe any
  patents or other property right claims or to contest validity of any
  such claims; this section has the sole purpose of protecting the
  integrity of the free software distribution system which is
  implemented by public license practices.  Many people have made
  generous contributions to the wide range of software distributed
  through that system in reliance on consistent application of that
  system; it is up to the author/donor to decide if he or she is
  willing to distribute software through any other system and a
  licensee cannot impose that choice.
    
  This section is intended to make thoroughly clear what is believed
  to be a consequence of the rest of this License.
  
\item [12.] If the distribution and/or use of the Library is
  restricted in certain countries either by patents or by copyrighted
  interfaces, the original copyright holder who places the Library
  under this License may add an explicit geographical distribution
  limitation excluding those countries, so that distribution is
  permitted only in or among countries not thus excluded.  In such
  case, this License incorporates the limitation as if written in the
  body of this License.
    
\item [13.] The Free Software Foundation may publish revised and/or
  new versions of the Lesser General Public License from time to time.
  Such new versions will be similar in spirit to the present version,
  but may differ in detail to address new problems or concerns.
    
  Each version is given a distinguishing version number.  If the
  Library specifies a version number of this License which applies to
  it and "any later version", you have the option of following the
  terms and conditions either of that version or of any later version
  published by the Free Software Foundation.  If the Library does not
  specify a license version number, you may choose any version ever
  published by the Free Software Foundation.
  
\item [14.] If you wish to incorporate parts of the Library into other
  free programs whose distribution conditions are incompatible with
  these, write to the author to ask for permission.  For software
  which is copyrighted by the Free Software Foundation, write to the
  Free Software Foundation; we sometimes make exceptions for this.
  Our decision will be guided by the two goals of preserving the free
  status of all derivatives of our free software and of promoting the
  sharing and reuse of software generally.
    
\textbf{NO WARRANTY}

\item [15.] BECAUSE THE LIBRARY IS LICENSED FREE OF CHARGE, THERE IS
  NO WARRANTY FOR THE LIBRARY, TO THE EXTENT PERMITTED BY APPLICABLE
  LAW.  EXCEPT WHEN OTHERWISE STATED IN WRITING THE COPYRIGHT HOLDERS
  AND/OR OTHER PARTIES PROVIDE THE LIBRARY "AS IS" WITHOUT WARRANTY OF
  ANY KIND, EITHER EXPRESSED OR IMPLIED, INCLUDING, BUT NOT LIMITED
  TO, THE IMPLIED WARRANTIES OF MERCHANTABILITY AND FITNESS FOR A
  PARTICULAR PURPOSE.  THE ENTIRE RISK AS TO THE QUALITY AND
  PERFORMANCE OF THE LIBRARY IS WITH YOU.  SHOULD THE LIBRARY PROVE
  DEFECTIVE, YOU ASSUME THE COST OF ALL NECESSARY SERVICING, REPAIR OR
  CORRECTION.
    
\item [16.] IN NO EVENT UNLESS REQUIRED BY APPLICABLE LAW OR AGREED TO
  IN WRITING WILL ANY COPYRIGHT HOLDER, OR ANY OTHER PARTY WHO MAY
  MODIFY AND/OR REDISTRIBUTE THE LIBRARY AS PERMITTED ABOVE, BE LIABLE
  TO YOU FOR DAMAGES, INCLUDING ANY GENERAL, SPECIAL, INCIDENTAL OR
  CONSEQUENTIAL DAMAGES ARISING OUT OF THE USE OR INABILITY TO USE THE
  LIBRARY (INCLUDING BUT NOT LIMITED TO LOSS OF DATA OR DATA BEING
  RENDERED INACCURATE OR LOSSES SUSTAINED BY YOU OR THIRD PARTIES OR A
  FAILURE OF THE LIBRARY TO OPERATE WITH ANY OTHER SOFTWARE), EVEN IF
  SUCH HOLDER OR OTHER PARTY HAS BEEN ADVISED OF THE POSSIBILITY OF
  SUCH DAMAGES.
\end{itemize}

END OF TERMS AND CONDITIONS

}

\subsubsection*{How to Apply These Terms to Your New Libraries}

If you develop a new library, and you want it to be of the greatest
possible use to the public, we recommend making it free software that
everyone can redistribute and change.  You can do so by permitting
redistribution under these terms (or, alternatively, under the terms
of the ordinary General Public License).

To apply these terms, attach the following notices to the library.  It
is safest to attach them to the start of each source file to most
effectively convey the exclusion of warranty; and each file should
have at least the "copyright" line and a pointer to where the full
notice is found.

{\small
\begin{verbatim}

<one line to give the library's name and a brief idea of what it does.>
Copyright (C) <year>  <name of author>

This library is free software; you can redistribute it and/or
modify it under the terms of the GNU Lesser General Public
License as published by the Free Software Foundation; either
version 2.1 of the License, or (at your option) any later version.

This library is distributed in the hope that it will be useful,
but WITHOUT ANY WARRANTY; without even the implied warranty of
MERCHANTABILITY or FITNESS FOR A PARTICULAR PURPOSE.  See the GNU
Lesser General Public License for more details.

You should have received a copy of the GNU Lesser General Public
License along with this library; if not, write to the Free Software
Foundation, Inc., 59 Temple Place, Suite 330, Boston, MA  02111-1307  USA
\end{verbatim}
}

Also add information on how to contact you by electronic and paper mail.

You should also get your employer (if you work as a programmer) or your
school, if any, to sign a "copyright disclaimer" for the library, if
necessary.  Here is a sample; alter the names:

{\small
\begin{verbatim}

Yoyodyne, Inc., hereby disclaims all copyright interest in the
library `Frob' (a library for tweaking knobs) written by James Random Hacker.

<signature of Ty Coon>, 1 April 1990
Ty Coon, President of Vice
\end{verbatim}
}

That's all there is to it!




\end{document}
\subsection*{}
\subsubsection*{Syntax}
\begin{verbatim}
\end{verbatim}
\subsubsection*{Description}

{\bf Note: } 

\subsubsection*{Return Value}

\newpage

% LocalWords:  ident
